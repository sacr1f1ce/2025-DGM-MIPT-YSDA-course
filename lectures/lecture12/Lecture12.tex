\documentclass{beamer}
\usepackage[utf8]{inputenc}
\usepackage{graphicx, epsfig}
\usepackage{amsmath,mathrsfs,amsfonts,amssymb}
\usepackage{floatflt}
\usepackage{epic,ecltree}
\usepackage{mathtext}
\usepackage{fancybox}
\usepackage{fancyhdr}
\usepackage{multirow}
\usepackage{enumerate}
\usepackage{epstopdf}
\usepackage{multicol}
\usepackage{algorithm}
\usepackage[noend]{algorithmic}
\usepackage{tikz}
\usepackage{blindtext}
\usepackage{multido}
\usetheme{default}%{Singapore}%{Warsaw}%{Warsaw}%{Darmstadt}
\usecolortheme{default}

\setbeamerfont{title}{size=\Huge}
\setbeamertemplate{footline}[frame number]{}

\setbeamertemplate{section in toc}[sections numbered]

\makeatletter
\newcommand\HUGE{\@setfontsize\Huge{35}{40}}
\makeatother    

\setbeamerfont{title}{size=\HUGE}
\beamertemplatenavigationsymbolsempty

\usetikzlibrary{arrows,shapes,positioning,shadows,trees}

\newcommand\myfootnote[1]{%
  \vspace{-0.5cm}%
  \tikz[remember picture,overlay]
  \draw (current page.south west) +(1in + \oddsidemargin,0.5em)
  node[anchor=south west,inner sep=0pt]{\parbox{\textwidth}{%
      \rlap{\rule{10em}{0.4pt}}\raggedright\scriptsize \textit{#1}}};}

\newcommand\myfootnotewithlink[2]{%
  \vspace{-0.5cm}%
  \tikz[remember picture,overlay]
  \draw (current page.south west) +(1in + \oddsidemargin,0.5em)
  node[anchor=south west,inner sep=0pt]{\parbox{\textwidth}{%
      \rlap{\rule{10em}{0.4pt}}\raggedright\scriptsize\href{#1}{\textit{#2}}}};}

\AtBeginSection[]
      {
      	\begin{frame}{Outline}
      		\tableofcontents[currentsection]
      	\end{frame}
      }
      \AtBeginSubsection[]{
      	\begin{frame}{Outline}
      		\tableofcontents[currentsection,currentsubsection]
      	\end{frame}
}

\newcounter{noscounter}
\newcounter{pcounter}
\newcommand{\nextonslide}[1]{%
  \stepcounter{noscounter}%
  \stepcounter{pcounter}%
  \onslide<\value{noscounter}->{#1}%
}
\newcommand{\resetonslide}{%
    \setcounter{noscounter}{1}%
    \setcounter{pcounter}{0}%
}

\newcommand{\eqpause}{%
  \multido{\i=1+1}{\value{pcounter}}{\pause}%
  \stepcounter{noscounter}%
  \setcounter{pcounter}{0}%
  \pause%
}
\addtobeamertemplate{frametitle}{\resetonslide}{}

\input{../utils/newcommands}
\input{../utils/title}
\createdgmtitle{12}

\usepackage{tikz}

\usetikzlibrary{arrows,shapes,positioning,shadows,trees}
%--------------------------------------------------------------------------------
\begin{document}
%--------------------------------------------------------------------------------
\begin{frame}[noframenumbering,plain]
\titlepage
\end{frame}
%=======
\begin{frame}{Recap of Previous Lecture}
    \myfootnotewithlink{https://arxiv.org/abs/1806.07366}{Chen R. T. Q. et al. Neural Ordinary Differential Equations, 2018}   
	\[
 		\frac{d \bx(t)}{dt} = \bff_{\btheta}(\bx(t), t); \quad \text{with initial condition }\bx(t_0) = \bx_0
	\]
	\vspace{-0.3cm}
	\begin{block}{Theorem (Continuity Equation)}
		If $\bff$ is uniformly Lipschitz continuous in $\bx$ and continuous in $t$, then
		\[
			\frac{d \log p_t(\bx(t))}{d t} = - \text{tr} \left( \frac{\partial \bff(\bx(t), t)}{\partial \bx(t)} \right)
		\]
		\vspace{-0.5cm}
	\end{block}
	\begin{block}{Solution of the Continuity Equation}
		\vspace{-0.3cm}
		\[
			\log p_1(\bx(1)) = \log p_0(\bx(0)) - {\color{teal}\int_{0}^{1} \text{tr}  \left( \frac{\partial \bff(\bx(t), t)}{\partial \bx(t)} \right) dt}.
		\]
	\end{block}
	\begin{itemize}
		\item This solution gives us the density along the trajectory (not the total probability path).
		\item However, it's difficult to efficiently estimate {\color{teal}the last term}.
	 \end{itemize}
\end{frame}
%=======
\begin{frame}{Recap of Previous Lecture}
	\vspace{-0.2cm}
	\begin{block}{SDE Basics}
		Let's define a stochastic process $\bx(t)$ with initial condition $\bx(0) \sim p_0(\bx)$:
		\[
			d\bx = \bff(\bx, t) dt + g(t) d \bw, 
		\]
		where $\bw(t)$ is the standard Wiener process (Brownian motion):
		\vspace{-0.2cm}
		\[		
			\bw(t) - \bw(s) \sim \cN(0, (t - s) \bI), \quad d \bw = \bepsilon \cdot \sqrt{dt}, \, \text{where } \bepsilon \sim \cN(0, \bI).
		\]
	\end{block}
	\vspace{-0.3cm}
	\begin{block}{Discretization of SDE (Euler Method) - \texttt{SDESolve}}
		\vspace{-0.3cm}
		\[
			\bx(t + dt) = \bx(t) + \bff(\bx(t), t) \cdot dt + g(t) \cdot \bepsilon \cdot \sqrt{dt}
		\]
		\vspace{-0.3cm}
	\end{block}
	\begin{itemize}
		\item At each time $t$, we have the density $p_t(\bx) = p(\bx, t)$.
		\item $p: \bbR^m \times [0, 1] \rightarrow \bbR_+$ is a \textbf{probability path} between $p_0(\bx)$ and $p_1(\bx)$.
	\end{itemize}
\end{frame}
%=======
\begin{frame}{Recap of Previous Lecture}
    \myfootnotewithlink{https://www.stats.ox.ac.uk/~teh/research/compstats/WelTeh2011a.pdf}{Welling M. Bayesian Learning via Stochastic Gradient Langevin Dynamics, 2011}
 	\begin{block}{Theorem (Kolmogorov-Fokker-Planck)}
 		The evolution of the distribution $p_t(\bx)$ is given by:
 		\vspace{-0.3cm}
 		\[
 			\frac{\partial p_t(\bx)}{\partial t} = - \text{div}\left(\bff(\bx, t) p_t(\bx)\right) + \frac{1}{2}g^2(t) \Delta_{\bx}p_t(\bx)
 		\]
 		\vspace{-0.5cm}
 	\end{block}
 	\begin{block}{Langevin SDE (Special Case)}
 		\vspace{-0.3cm}
 		\[
 			d \bx = {\color{violet}\frac{1}{2} \frac{\partial}{\partial \bx} \log p_t(\bx)} d t + {\color{teal} 1 } \cdot d \bw
 		\]
 		\vspace{-0.3cm}
 	\end{block}
 	The density $p(\bx | \btheta)$ is a \textbf{stationary} distribution for the SDE.
	\begin{block}{Langevin Dynamics}
		Samples from the following dynamics will come from $p(\bx | \btheta)$ under mild regularity conditions for a small enough $\eta$:
		\vspace{-0.2cm}
		\[
			\bx_{t + 1} = \bx_t + \frac{\eta}{2} \nabla_{\bx_t} \log p(\bx_t | \btheta) + \sqrt{\eta} \cdot \bepsilon, \quad \bepsilon \sim \cN(0, \bI).
		\]
	\end{block}
\end{frame}
%=======
\begin{frame}{Recap of Previous Lecture}
	\myfootnotewithlink{https://arxiv.org/abs/2011.13456}{Song Y., et al. Score-Based Generative Modeling through Stochastic Differential Equations, 2020}
	\vspace{-0.3cm}
	\[
		d\bx = \bff(\bx, t) dt + g(t) d \bw \qquad \text{(SDE with the probability path $p_t(\bx)$)}
	\]
	\vspace{-0.7cm}
	\begin{block}{Probability Flow ODE}
		There exists an ODE with the identical probability path $p_t(\bx)$ of the form:
		\vspace{-0.3cm}
		\[
			d\bx = \left(\bff(\bx, t) -\frac{1}{2} g^2(t) \frac{\partial}{\partial \bx} \log p_t(\bx) \right) dt
		\]
		\vspace{-0.7cm}
	\end{block}
	\begin{figure}
		\includegraphics[width=0.75\linewidth]{figs/probability_flow}
	\end{figure}
\end{frame}
%=======
\begin{frame}{Recap of Previous Lecture}
	\myfootnotewithlink{https://arxiv.org/abs/2011.13456}{Song Y., et al. Score-Based Generative Modeling through Stochastic Differential Equations, 2020}
	\[
		d\bx = \bff(\bx, t) dt, \quad \bx(t + dt) = \bx(t) + \bff(\bx, t) dt
	\]
	\vspace{-0.7cm}
	\begin{block}{Reverse ODE}
		Let $\tau = 1 - t$ ($d\tau = -dt$):
		\vspace{-0.3cm}
		\[
			d\bx = - \bff(\bx, 1 - \tau) d \tau
		\]
	\end{block}
	\vspace{-0.5cm}
	\begin{block}{Reverse SDE}
		There's a reverse SDE for $d\bx = \bff(\bx, t) dt + g(t) d\bw$ in the following form:
		\vspace{-0.3cm}
		\[
			d\bx = \left(\bff(\bx, t) {\color{violet}- g^2(t) \frac{\partial}{\partial \bx}\log p_t(\bx)}\right) dt + g(t) d \bw, \quad dt < 0
		\] 
	\end{block}
	\vspace{-0.5cm}
	\begin{block}{Sketch of the Proof}
		\begin{itemize}
			\item Convert the initial SDE to the probability flow ODE.
			\item Reverse the probability flow ODE.
			\item Convert the reverse probability flow ODE to the reverse SDE.
		\end{itemize}
	\end{block}
\end{frame}
%=======
\begin{frame}{Recap of Previous Lecture}
	\myfootnotewithlink{https://arxiv.org/abs/2011.13456}{Song Y., et al. Score-Based Generative Modeling through Stochastic Differential Equations, 2020}
	\vspace{-0.5cm}
	\begin{align*}
		d\bx &= \bff(\bx, t) dt + g(t) d \bw \qquad\quad\;\:\, \text{(SDE)} \\
		d\bx &= \left(\bff(\bx, t) -\frac{1}{2} g^2(t) \frac{\partial}{\partial \bx} \log p_t(\bx) \right) dt \;\,\,\, \text{(probability flow ODE)} \\
		d\bx &= \left(\bff(\bx, t) - g^2(t) \frac{\partial }{\partial \bx}\log p_t(\bx)\right) dt + g(t) d \bw \quad \text{(reverse SDE)}
	\end{align*}
	\vspace{-0.5cm}
	\begin{figure}
		\includegraphics[width=\linewidth]{figs/sde}
	\end{figure}
\end{frame}
%=======
\begin{frame}{Outline}
	\tableofcontents
\end{frame}
%=======
\section{Diffusion and Score Matching SDEs}
%=======
\begin{frame}{Score Matching SDE}
	\myfootnotewithlink{https://arxiv.org/abs/2011.13456}{Song Y., et al. Score-Based Generative Modeling through Stochastic Differential Equations, 2020}
	\vspace{-0.3cm}
	\begin{block}{Denoising Score Matching}
		\vspace{-0.7cm}
		\begin{align*}
			\bx_t &= \bx + \sigma_t \cdot \bepsilon_t, & q(\bx_t | \bx) &= \cN(\bx, \sigma_t^2 \cdot \bI) \\
			\bx_{t-1} &= \bx + \sigma_{t-1} \cdot \bepsilon_{t-1}, & q(\bx_{t-1} | \bx) &= \cN(\bx, \sigma_{t-1}^2 \cdot \bI)
		\end{align*}
	\end{block}
	\vspace{-1.0cm}
	\[
		\bx_t = \bx_{t - 1} + \sqrt{\sigma^2_t - \sigma^2_{t-1}} \cdot \bepsilon, \quad q(\bx_{t} | \bx_{t-1}) = \cN(\bx_{t-1}, (\sigma_t^2 - \sigma_{t-1}^2) \cdot \bI)
	\]
	Let's transform this Markov chain into the continuous stochastic process~$\bx(t)$ by letting $T \rightarrow \infty$:
	\vspace{-0.3cm}
	\begin{align*}
		\bx(t) &= \bx(t - dt) + \sqrt{\sigma^2(t) - \sigma^2(t - dt)} \cdot {\color{violet}\bepsilon} \\
		&= \bx(t - dt) + \sqrt{\frac{\sigma^2(t) - \sigma^2(t - dt)}{dt} {\color{violet}dt}} \cdot {\color{violet}\bepsilon} \\
		&= \bx(t - dt) + \sqrt{\frac{ d [\sigma^2(t)]}{dt}} \cdot {\color{violet}d \bw}
	\end{align*}
	\vspace{-0.5cm}
\end{frame}
%=======
\begin{frame}{Score Matching SDE}
	\myfootnotewithlink{https://arxiv.org/abs/2011.13456}{Song Y., et al. Score-Based Generative Modeling through Stochastic Differential Equations, 2020}
	\vspace{-0.3cm}
	\[
		\bx(t) = \bx(t - dt) + \sqrt{\frac{ d [\sigma^2(t)]}{dt}} \cdot d \bw
	\]
	\vspace{-0.5cm}
	\begin{block}{Variance Exploding SDE}
		\vspace{-0.3cm}
		\[
			d \bx = \sqrt{\frac{ d [\sigma^2(t)]}{dt}} \cdot d \bw
		\]
		$\sigma(t)$ is a monotonically increasing function.
	\end{block}
	\vspace{-0.7cm}
	\[
		d\bx = \bff(\bx, t) dt + g(t) d \bw, \quad \bff(\bx, t) = 0, \quad g(t) = \sqrt{\frac{ d [\sigma^2(t)]}{dt}} 
	\]
	\vspace{-0.5cm}
	\begin{align*}
		d\bx &= \left(-\frac{1}{2} \frac{ d [\sigma^2(t)]}{dt} \frac{\partial}{\partial \bx} \log p_t(\bx) \right) dt \qquad\quad\, \text{(probability flow ODE)} \\
		d\bx &= \left(- \frac{ d [\sigma^2(t)]}{dt} \frac{\partial}{\partial \bx}\log p_t(\bx)\right) dt + \sqrt{\frac{ d [\sigma^2(t)]}{dt}}  d \bw \ \text{(reverse SDE)}
	\end{align*}
\end{frame}
%=======
\begin{frame}{Diffusion SDE}
    \myfootnotewithlink{https://arxiv.org/abs/2011.13456}{Song Y., et al. Score-Based Generative Modeling through Stochastic Differential Equations, 2020}
	\begin{block}{Denoising Diffusion}
		\vspace{-0.7cm}
		\[
			\bx_t = \sqrt{1 - \beta_t} \cdot \bx_{t - 1} + \sqrt{\beta_t} \cdot \bepsilon, \quad q(\bx_t | \bx_{t-1}) = \cN(\sqrt{1 - \beta_t} \cdot \bx_{t-1}, \beta_t \cdot \bI)
		\]
		\vspace{-0.7cm}
	\end{block}
	Let's turn this Markov chain into a continuous stochastic process by letting $T \rightarrow \infty$ and setting $\beta_t = \beta(\frac{t}{T}) \cdot \frac{1}{T}$ (where $dt = \frac{1}{T}$):
	\begin{multline*}
		{\color{teal}\bx(t)} = \sqrt{1 - \beta(t) dt} \cdot \bx(t - dt) + \sqrt{\beta(t)dt} \cdot \bepsilon \approx \\
		\approx (1 - \frac{1}{2} \beta(t) dt) \cdot \bx(t - dt) + \sqrt{\beta(t){\color{violet}dt}} \cdot {\color{violet}\bepsilon} = \\
		= {\color{teal}\bx(t - dt)} - \frac{1}{2} \beta(t) \bx(t - dt) dt  + \sqrt{\beta(t)} \cdot {\color{violet}d \bw}
	\end{multline*}
	\vspace{-0.5cm}
	\begin{block}{Variance Preserving SDE}
		\vspace{-0.3cm}
		\[
			{\color{teal}d \bx} = - \frac{1}{2} \beta(t) \bx(t) dt + \sqrt{\beta(t)} \cdot d \bw
		\]
	\end{block}
\end{frame}
%=======
\begin{frame}{Diffusion SDE}
	\myfootnotewithlink{https://arxiv.org/abs/2011.13456}{Song Y., et al. Score-Based Generative Modeling through Stochastic Differential Equations, 2020}
	\begin{block}{Variance Preserving SDE}
		\vspace{-0.3cm}
		\[
			d \bx = - \frac{1}{2} \beta(t) \bx(t) dt + \sqrt{\beta(t)} \cdot d \bw
		\]
		\[
			\bff(\bx, t) = - \frac{1}{2} \beta(t) \bx(t) , \quad g(t) = \sqrt{\beta(t)} 
		\]
	\end{block}
	Variance is preserved as long as $\bx(0)$ has unit variance.
	\begin{align*}
		d\bx &= \left(- \frac{1}{2} \beta(t) \bx(t) - \frac{1}{2} \beta(t) \frac{\partial}{\partial \bx} \log p_t(\bx) \right) dt \quad\quad \text{(probability flow ODE)} \\
		d\bx &= \left(- \frac{1}{2} \beta(t) \bx(t) - \beta(t) \frac{\partial}{\partial \bx} \log p_t(\bx)\right) dt + \sqrt{\beta(t)} d \bw\ \ \text{(reverse SDE)}
	\end{align*}
\end{frame}
%=======
\begin{frame}{Diffusion SDE}
	\myfootnotewithlink{https://arxiv.org/abs/2206.00927}{Lu C. et al. Dpm-solver: A fast ode solver for diffusion probabilistic model sampling in around 10 steps, 2022}
	\vspace{-0.5cm}
	\[
		d\bx = {\color{teal}\bff(\bx, t)} dt + {\color{violet}g(t)} d \bw
	\]
	\vspace{-0.5cm}
	\begin{block}{Variance Exploding SDE (NCSN)}
		\vspace{-0.3cm}
		\[
			d \bx = {\color{violet}\sqrt{\frac{ d [\sigma^2(t)]}{dt}}} \cdot d \bw
		\]
		\vspace{-0.5cm}
	\end{block}
	\begin{block}{Variance Preserving SDE (DDPM)}
		\vspace{-0.3cm}
		\[
			d \bx = {\color{teal}- \frac{1}{2} \beta(t) \bx(t)} dt + {\color{violet}\sqrt{\beta(t)}} \cdot d \bw
		\]
		\vspace{-0.5cm}
	\end{block}
	\begin{block}{Efficient Solvers}
		\begin{itemize}
		\item Converting SDEs to PF-ODEs yields more efficient inference.
		\item We can apply any \texttt{ODESolve} procedure to reduce the number of inference steps.
		\item In practice, this reduces the number of steps from $100$--$1000$ to $20$--$50$.
		\end{itemize}
	\end{block}
\end{frame}
%=======
\section{Score-Based Generative Models Through SDEs}
%=======
\begin{frame}{Score-Based Generative Models Through SDEs}
    \myfootnotewithlink{https://arxiv.org/abs/2011.13456}{Song Y., et al. Score-Based Generative Modeling through Stochastic Differential Equations, 2020}
	\begin{block}{Discrete-Time Objective}
		\vspace{-0.3cm}
		\[
			\bbE_{\pi(\bx_0)} \bbE_{t \sim U\{1, T\}} \bbE_{q(\bx_t | \bx_0)}\bigl\| \bs_{\btheta, t}(\bx_t) - \nabla_{\bx_t} \log q(\bx_t | \bx_0) \bigr\|^2_2 
		\]
		\vspace{-0.5cm}
	\end{block}
	Is it possible to train score-based diffusion models in continuous time?
	\begin{block}{Continuous-Time Objective}
		\vspace{-0.7cm}
		\[
			\bbE_{\pi(\bx(0))} \bbE_{t \sim U[0, 1]} \bbE_{q(\bx(t) | \bx(0))}\bigl\| \bs_{\btheta}(\bx(t), t) - {\color{teal}\nabla_{\bx(t)} \log q(\bx(t) | \bx(0))} \bigr\|^2_2 
		\]
		\vspace{-0.7cm}
	\end{block}
	\vspace{-0.4cm}
	\begin{figure}
		\includegraphics[width=0.75\linewidth]{figs/sbgm}
	\end{figure}
\end{frame}
%=======
\begin{frame}{Score-Based Generative Models Through SDEs}
	\myfootnotewithlink{https://users.aalto.fi/~asolin/sde-book/sde-book.pdf}{Särkkä S., Solin A. Applied stochastic differential equations, 2019}
	\vspace{-0.3cm}
	\begin{block}{Continuous-Time Objective}
		\vspace{-0.5cm}
		\[
			\bbE_{\pi(\bx(0))} \bbE_{t \sim U[0, 1]} \bbE_{q(\bx(t) | \bx(0))}\bigl\| \bs_{\btheta}(\bx(t), t) - {\color{teal}\nabla_{\bx(t)} \log q(\bx(t) | \bx(0))} \bigr\|^2_2 
		\]
		\vspace{-0.7cm}
	\end{block}
	\[
		q(\bx(t) | \bx(0)) = \cN\Bigl(\bmu(\bx(t), \bx(0)), \bsigma^2(\bx(t), \bx(0)) \cdot \bI \Bigr)
	\]
	\[
		\nabla_{\bx(t)} \log q(\bx(t) | \bx(0)) = - \frac{1}{\bsigma} \odot (\bx(t) - \bmu)
	\]
	\[
		d \bx = \sqrt{\frac{ d [\sigma^2(t)]}{dt}} \cdot d \bw \ \ \mbox{(Variance Exploding SDE)}
	\]
	\vspace{-0.3cm}
	\[
		d \bx = - \frac{1}{2} \beta(t) \bx(t) dt + \sqrt{\beta(t)} \cdot d \bw \ \ \mbox{(Variance Preserving SDE)}
	\]
	Is it possible to explicitly derive $\bmu(\bx(t), \bx(0))$ and $\bSigma(\bx(t), \bx(0))$ for VE-SDE and VP-SDE?
\end{frame}
%=======
\begin{frame}{Score-Based Generative Models Through SDEs}
	\myfootnotewithlink{https://users.aalto.fi/~asolin/sde-book/sde-book.pdf}{Särkkä S., Solin A. Applied stochastic differential equations, 2019}
	\[
		q(\bx(t) | \bx(0)) = \cN\Bigl(\bmu(\bx(t), \bx(0)), \bSigma(\bx(t), \bx(0))\Bigr)
	\]
	\vspace{-0.5cm}
	\begin{block}{Theorem}
		The moments of the SDE $d\bx = \bff(\bx, t) dt + g(t) d \bw$ satisfy:
		\[
			\frac{d \bmu(\bx(t), \bx(0))}{dt} = \bbE \left[ \bff(\bx(t), t) | \bx(0)\right]
		\]
		\[
			\frac{d \bSigma(\bx(t), \bx(0))}{dt} = \bbE \left[ \bff \cdot (\bx(t) - \bmu)^T + (\bx(t) - \bmu) \cdot \bff^T | \bx(0)\right] + g^2(t) \cdot \bI
		\]
	\end{block}
	\begin{block}{Proof}
		\vspace{-0.7cm}
		\begin{align*}
			\bbE\left[{\color{violet}d\bx} | \bx(0) \right] &= \bbE\left[{\color{violet}\bff(\bx, t) dt} | \bx(0) \right] + \bbE\left[{\color{violet}g(t) d \bw} | \bx(0) \right] \\
			&= \bbE\left[\bff(\bx, t) | \bx(0) \right] dt + g(t) \bbE\left[d \bw | \bx(0) \right] \\
			&= \bbE\left[\bff(\bx, t) | \bx(0) \right] dt
		\end{align*}
	\end{block}
\end{frame}
%=======
\begin{frame}{Score-Based Generative Models Through SDEs}
	\myfootnotewithlink{https://users.aalto.fi/~asolin/sde-book/sde-book.pdf}{Särkkä S., Solin A. Applied stochastic differential equations, 2019}
	\begin{block}{Theorem}
		\vspace{-0.3cm}
		\[
			\frac{d \bmu(\bx(t), \bx(0))}{dt} = \bbE \left[ \bff(\bx(t), t) | \bx(0)\right]
		\]
		\vspace{-0.5cm}
	\end{block}
	\begin{block}{Proof (Continued)}
		\vspace{-0.3cm}
		\[
			\bbE\left[d\bx | \bx(0) \right] = \bbE\left[\bff(\bx, t) | \bx(0) \right] dt
		\]
		\[
			\frac{d \bbE\left[\bx(t) | \bx(0) \right]}{dt} = \frac{d \bmu(\bx(t), \bx(0))}{dt} = \bbE\left[\bff(\bx, t) | \bx(0) \right] 
		\]
	\end{block}
	\vspace{-0.3cm}
	\begin{block}{Examples}
		\vspace{-0.5cm}
		\[
			\textbf{NCSN:}\quad	\bff(\bx, t) = 0 \quad \Rightarrow \quad \bmu = \bx(0)
		\]
		\[
			\textbf{DDPM:}\quad \bff(\bx, t) = - \frac{1}{2} \beta(t) \bx(t)\;\;   \Rightarrow\quad \frac{d \bmu}{dt} = - \frac{1}{2} \beta(t) \bmu
		\]
		\[
			\bmu = \bx(0) \exp\left(- \frac{1}{2} \int_0^t \beta(s)ds\right)
		\]
	\end{block}
\end{frame}
%=======
\begin{frame}{Score-Based Generative Models Through SDEs}
	\myfootnotewithlink{https://arxiv.org/abs/2011.13456}{Song Y., et al. Score-Based Generative Modeling through Stochastic Differential Equations, 2020}
	\begin{block}{Training}
		\vspace{-0.5cm}
		\[
			\bbE_{\pi(\bx(0))} \bbE_{t \sim U[0, 1]} \bbE_{q(\bx(t) | \bx(0))}\bigl\| \bs_{\btheta}(\bx(t), t) - {\color{teal}\nabla_{\bx(t)} \log q(\bx(t) | \bx(0))} \bigr\|^2_2 
		\]
		\vspace{-0.7cm}
	\end{block}
	\[
		q(\bx(t) | \bx(0)) = \cN\Bigl(\bmu(\bx(t), \bx(0)), \bSigma(\bx(t), \bx(0))\Bigr)
	\]
	\vspace{-0.5cm}
	\begin{block}{NCSN}
		\vspace{-0.3cm}
		\[
			q(\bx(t) | \bx(0)) = \cN\left(\bx(0), \left[\sigma^2(t) - \sigma^2(0)\right] \cdot \bI\right)
		\]
		\vspace{-0.5cm}
	\end{block}
	\begin{block}{DDPM}
		\vspace{-0.3cm}
		\[
			q(\bx(t) | \bx(0)) = \cN\left(\bx(0) e^{-\frac{1}{2} \int_0^t\beta(s)ds}, \left(1 - e^{- \int_0^t\beta(s)ds}\right) \cdot \bI\right)
		\]
		\vspace{-0.5cm}
	\end{block}
	Here we omit the derivations of the variance.
	
\end{frame}
%=======
\begin{frame}{Score-Based Generative Models Through SDEs}
	\myfootnotewithlink{https://arxiv.org/abs/2011.13456}{Song Y., et al. Score-Based Generative Modeling through Stochastic Differential Equations, 2020}
	\begin{block}{Sampling}
		Solve the reverse SDE using numerical solvers (\texttt{SDESolve}).
		\begin{figure}
			\includegraphics[width=0.8\linewidth]{figs/sbgm}
		\end{figure}
		\vspace{-0.5cm}
	\end{block}
	\begin{itemize}
		\item Discretizing the reverse SDE provides ancestral sampling.
		\item Discretizing the probability flow ODE yields deterministic sampling.
	\end{itemize}
\end{frame}
%=======
\section{Flow Matching}
%=======
\begin{frame}{Continuous-Time Normalizing Flows}
    \myfootnotewithlink{https://arxiv.org/abs/1806.07366}{Chen R. T. Q. et al. Neural Ordinary Differential Equations, 2018}   

	Let's return to ODE dynamics $\bx(t)$ in the interval $t \in [0, 1]$:
	\begin{itemize}
	\item $\bx_0 \sim p_0(\bx) = p(\bx)$, $\bx_1 \sim p_1(\bx) =  \pi(\bx)$;
	\item $p(\bx)$ is a base distribution (e.g., $\cN(0, \bI)$), and $\pi(\bx)$ is the true data distribution.
	\end{itemize}
	\[
		\frac{d \bx}{dt} = \bff (\bx, t),  \quad \text{with initial condition } \bx(0) = \bx_0.
	\]
	\vspace{-0.3cm}
	\begin{block}{KFP Theorem (Continuity Equation)}
		\vspace{-0.5cm}
		\[
			\frac{\partial p_t(\bx)}{\partial t} = - \text{div}\left(\bff(\bx, t) p_t(\bx)\right) \Leftrightarrow \frac{d \log p_t(\bx(t))}{d t} = - \text{tr} \left( \frac{\partial \bff(\bx(t), t)}{\partial \bx(t)} \right)
		\]
		\vspace{-0.3cm}
	\end{block}
	\begin{itemize}
		\item It's hard to solve the continuity equation directly due to the trace term.
		\item There's a method (the adjoint method) that solves this equation directly, but it's unstable and unscalable.
	\end{itemize}
\end{frame}
%=======
\begin{frame}{Continuous-Time Normalizing Flows}
    \myfootnotewithlink{https://arxiv.org/abs/2210.02747}{Lipman Y., et al. Flow Matching for Generative Modeling, 2022}
	\begin{block}{KFP Theorem (Continuity Equation)}
		\vspace{-0.5cm}
		\[
			\frac{\partial p_t(\bx)}{\partial t} = - \text{div}\left(\bff(\bx, t) p_t(\bx)\right) \Leftrightarrow \frac{d \log p_t(\bx(t))}{d t} = - \text{tr} \left( \frac{\partial \bff(\bx(t), t)}{\partial \bx(t)} \right)
		\]
		\vspace{-0.3cm}
	\end{block}
	\begin{itemize}
		\item Knowing the vector field $\bff (\bx, t)$, the KFP (or continuity) equation allows us to compute the density $p_t(\bx)$.
		\item Flow matching provides an alternative approach to Neural ODEs.
	\end{itemize}
	\begin{block}{Flow Matching}
		\vspace{-0.3cm}
		\[
			\bbE_{t \sim U[0, 1]} \bbE_{\bx \sim p_t(\bx)}\left\| \bff(\bx, t) - \bff_{\btheta}(\bx, t) \right\|^2 \rightarrow \min_{\btheta}
		\]
		\vspace{-0.3cm}
	\end{block}
	\begin{itemize}
		\item Approximate the true vector field $\bff (\bx, t)$ using $\bff_{\btheta}(\bx, t)$.
		\item Use $\bff_{\btheta}(\bx, t)$ for deterministic sampling from the ODE.
	\end{itemize}
\end{frame}
%=======
\begin{frame}{Flow Matching}
    \myfootnotewithlink{https://dl.heeere.com/conditional-flow-matching/blog/conditional-flow-matching}{image credit: A Visual Dive into Conditional Flow Matching}
	\[
		\bbE_{t \sim U[0, 1]} \bbE_{\bx \sim p_t(\bx)}\left\| \bff(\bx, t) - \bff_{\btheta}(\bx, t) \right\|^2 \rightarrow \min_{\btheta}
	\]
	\vspace{-0.5cm}
	\begin{itemize}
		\item There are infinitely many possible $\bff(\bx, t)$ between $\pi(\bx)$ and $p(\bx)$.
		\item The true vector field  $\bff(\bx, t)$ is \textbf{unknown}.
		\item We need to select the "best" $\bff(\bx, t)$ and make the objective tractable.
	\end{itemize}
	\begin{figure}
		\centering
		\includegraphics[width=\linewidth]{figs/multiple_dynamics}
	\end{figure}
\end{frame}
%=======
\section{Conditional Flow Matching}
%=======
\begin{frame}{Flow Matching}
    \myfootnotewithlink{https://arxiv.org/abs/2302.00482}{Tong A., et al. Improving and Generalizing Flow-Based Generative Models with Minibatch Optimal Transport, 2023}
	\vspace{-0.5cm}
	\begin{block}{Latent Variable Model}
		Let's introduce the latent variable $\bz$:
		\[
			p_t(\bx) = \int p_t(\bx | \bz) p(\bz) d \bz 
		\]
		\vspace{-0.5cm}
	\end{block}
	Here, $p_t(\bx | \bz)$ is a \textbf{conditional probability path}.
	
	The conditional probability path $p_t(\bx | \bz)$ satisfies the KFP theorem:
	\[
		{\color{violet}\frac{\partial p_t(\bx | \bz)}{\partial t} = - \text{div}\left(\bff(\bx, \bz, t) p_t(\bx | \bz)\right)},
	\]
	where $\bff(\bx, \bz, t)$ is a \textbf{conditional vector field}:
	\[
		\frac{d\bx}{dt} = \bff(\bx, t) \quad \Rightarrow \quad \frac{d\bx}{dt} = \bff(\bx, \bz, t)
	\]
	\vspace{-0.3cm}
	What's the relationship between $\bff(\bx, t)$ and $\bff(\bx, \bz, t)$?
\end{frame}
%=======
\begin{frame}{Flow Matching}
    \myfootnotewithlink{https://arxiv.org/abs/2302.00482}{Tong A., et al. Improving and Generalizing Flow-Based Generative Models with Minibatch Optimal Transport, 2023}
	\[
		{\color{violet}\frac{\partial p_t(\bx | \bz)}{\partial t} = - \text{div}\left(\bff(\bx, \bz, t) p_t(\bx | \bz)\right)},
	\]
	\vspace{-0.3cm}
	\begin{block}{Theorem}
		The following vector field generates the probability path $p_t(\bx)$:
		\vspace{-0.2cm}
		\[
			\bff(\bx, t) = \bbE_{p_t(\bz | \bx)} \bff(\bx, \bz, t)  = {\color{teal}\int \bff(\bx, \bz, t)} \frac{\color{teal}p_t(\bx | \bz) p(\bz)}{p_t(\bx)} {\color{teal}d \bz}
		\]
		\vspace{-0.5cm}
	\end{block}
	\begin{block}{Proof}
		\vspace{-0.7cm}
		\begin{multline*}
			\frac{\partial p_t(\bx)}{\partial t} = \frac{\partial}{\partial t} \int p_t(\bx | \bz) p(\bz) d \bz =  \int \left( {\color{violet} \frac{\partial p_t(\bx | \bz)}{\partial t} } \right) p(\bz) d \bz = \\
			= \int \left( {\color{violet} - \text{div}\left(\bff(\bx, \bz, t) p_t(\bx | \bz)\right) } \right) p(\bz) d \bz = \\
			= - \text{div} \left({\color{teal}\int \bff(\bx, \bz, t) p_t(\bx | \bz) p(\bz) d \bz }\right) = - \text{div}  \left(\bff(\bx, t) p_t(\bx)\right)
		\end{multline*}
	\end{block}
\end{frame}
%=======
\begin{frame}{Flow Matching}
    \myfootnotewithlink{https://arxiv.org/abs/2302.00482}{Tong A., et al. Improving and Generalizing Flow-Based Generative Models with Minibatch Optimal Transport, 2023}
	\begin{block}{Flow Matching (FM)}
		\vspace{-0.3cm}
		\[
			\bbE_{t \sim U[0, 1]} \bbE_{\bx \sim p_t(\bx)}\left\| \bff(\bx, t) - \bff_{\btheta}(\bx, t) \right\|^2 \rightarrow \min_{\btheta}
		\]
		\vspace{-0.3cm}
	\end{block}
	\begin{block}{Conditional Flow Matching (CFM)}
		\vspace{-0.3cm}
		\[
			\bbE_{t \sim U[0, 1]} \bbE_{\bz \sim p(\bz)} \bbE_{\bx \sim p_t(\bx | \bz)}\left\| \bff(\bx, \bz, t) - \bff_{\btheta}(\bx, t) \right\|^2 \rightarrow \min_{\btheta}
		\]
		\vspace{-0.3cm}
	\end{block}
	\begin{block}{Theorem}
		If $\text{supp}(p_t(\bx)) = \bbR^m$, then the optimal value of the FM objective equals the optimal value of the CFM objective.
	\end{block}
	\begin{block}{Proof}
		This can be proved in a similar way as in the denoising score matching theorem.
	\end{block}
\end{frame}
%=======
\begin{frame}{Conditional Flow Matching}
    \myfootnotewithlink{https://dl.heeere.com/conditional-flow-matching/blog/conditional-flow-matching}{image credit: A Visual Dive into Conditional Flow Matching}
	\begin{figure}
		\centering
		\includegraphics[width=0.75\linewidth]{figs/cfm_uncond_to_cond}
	\end{figure}
	\begin{itemize}
		\item We don't want to directly model $p_t(\bx)$, since it's complex.
		\item We've shown it's possible to solve the CFM task instead of the FM task.
		\item Let's choose a convenient conditioning latent variable $\bz$.
		\item We'll parametrize $p_t(\bx | \bz)$ instead of $p_t(\bx)$. It should satisfy the following constraints:
		\[
			p(\bx) = \cN(0, \bI) = \bbE_{p(\bz)} p_0(\bx | \bz); \quad \pi(\bx) = \bbE_{p(\bz)} p_1(\bx | \bz).
		\]
	\end{itemize}
\end{frame}
%=======
\begin{frame}{Summary}
	\begin{itemize}
		\item Score matching (NCSN) and diffusion models (DDPM) are discretizations of SDEs (variance exploding and variance preserving).
		\vfill
		\item It's possible to train continuous-in-time score-based generative models using forward and reverse SDEs.
		\vfill
		\item Discretizing the reverse SDE yields ancestral sampling of the DDPM.
		\vfill
		\item Flow matching suggests fitting the vector field directly.
		\vfill 
		\item Conditional flow matching introduces the latent variable $\bz$, reformulating the initial task in terms of conditional dynamics.
	\end{itemize}
\end{frame}
\end{document}