\documentclass{beamer}
\usepackage[utf8]{inputenc}
\usepackage{graphicx, epsfig}
\usepackage{amsmath,mathrsfs,amsfonts,amssymb}
\usepackage{floatflt}
\usepackage{epic,ecltree}
\usepackage{mathtext}
\usepackage{fancybox}
\usepackage{fancyhdr}
\usepackage{multirow}
\usepackage{enumerate}
\usepackage{epstopdf}
\usepackage{multicol}
\usepackage{algorithm}
\usepackage[noend]{algorithmic}
\usepackage{tikz}
\usepackage{blindtext}
\usepackage{multido}
\usetheme{default}%{Singapore}%{Warsaw}%{Warsaw}%{Darmstadt}
\usecolortheme{default}

\setbeamerfont{title}{size=\Huge}
\setbeamertemplate{footline}[frame number]{}

\setbeamertemplate{section in toc}[sections numbered]

\makeatletter
\newcommand\HUGE{\@setfontsize\Huge{35}{40}}
\makeatother    

\setbeamerfont{title}{size=\HUGE}
\beamertemplatenavigationsymbolsempty

\usetikzlibrary{arrows,shapes,positioning,shadows,trees}

\newcommand\myfootnote[1]{%
  \vspace{-0.5cm}%
  \tikz[remember picture,overlay]
  \draw (current page.south west) +(1in + \oddsidemargin,0.5em)
  node[anchor=south west,inner sep=0pt]{\parbox{\textwidth}{%
      \rlap{\rule{10em}{0.4pt}}\raggedright\scriptsize \textit{#1}}};}

\newcommand\myfootnotewithlink[2]{%
  \vspace{-0.5cm}%
  \tikz[remember picture,overlay]
  \draw (current page.south west) +(1in + \oddsidemargin,0.5em)
  node[anchor=south west,inner sep=0pt]{\parbox{\textwidth}{%
      \rlap{\rule{10em}{0.4pt}}\raggedright\scriptsize\href{#1}{\textit{#2}}}};}

\AtBeginSection[]
      {
      	\begin{frame}{Outline}
      		\tableofcontents[currentsection]
      	\end{frame}
      }
      \AtBeginSubsection[]{
      	\begin{frame}{Outline}
      		\tableofcontents[currentsection,currentsubsection]
      	\end{frame}
}

\newcounter{noscounter}
\newcounter{pcounter}
\newcommand{\nextonslide}[1]{%
  \stepcounter{noscounter}%
  \stepcounter{pcounter}%
  \onslide<\value{noscounter}->{#1}%
}
\newcommand{\resetonslide}{%
    \setcounter{noscounter}{1}%
    \setcounter{pcounter}{0}%
}

\newcommand{\eqpause}{%
  \multido{\i=1+1}{\value{pcounter}}{\pause}%
  \stepcounter{noscounter}%
  \setcounter{pcounter}{0}%
  \pause%
}
\addtobeamertemplate{frametitle}{\resetonslide}{}

\input{../utils/newcommands}
\input{../utils/title}
\createdgmtitle{11}
%--------------------------------------------------------------------------------
\begin{document}
%--------------------------------------------------------------------------------
\begin{frame}[noframenumbering,plain]
\titlepage
	\resetonslide
\end{frame}
%=======
\begin{frame}{Recap of Previous Lecture}
	\myfootnotewithlink{https://arxiv.org/abs/2006.11239}{Ho J. Denoising Diffusion Probabilistic Models, 2020}
	\begin{block}{DDPM Objective}
		\vspace{-0.5cm}
		\[
			\bbE_{\pd(\bx_0)} \bbE_{t \sim U\{1, T\}}\bbE_{q(\bx_t | \bx_0)} \left[ {\color{olive}\frac{(1 - \alpha_t)^2}{2\tilde{\beta}_t \alpha_t}} \Bigl\|  \bs_{\btheta, t} (\bx_t) - \nabla_{\bx_t} \log q(\bx_t | \bx_0) \Bigr\|_2^2  \right]
		\]
		In practice, {\color{olive}this coefficient} is usually omitted.
	\end{block}
	\begin{block}{NCSN Objective}
		\vspace{-0.3cm}
		\[
			\bbE_{\pd(\bx_0)} \bbE_{t \sim U\{1, T\}} \bbE_{q(\bx_t | \bx_0)}\bigl\| \bs_{\btheta, \sigma_t}(\bx_t) - \nabla_{\bx_t} \log q(\bx_t | \bx_0) \bigr\|^2_2 
		\]
		\vspace{-0.3cm}
	\end{block}
	\textbf{Note:} The objectives of DDPM and NCSN are almost identical; however, their sampling procedures differ:
	\begin{itemize}
		\item NCSN utilizes annealed Langevin dynamics,
		\item DDPM employs ancestral sampling.
	\end{itemize}
\end{frame}
%=======
\begin{frame}{Recap of Previous Lecture}
	\myfootnotewithlink{https://arxiv.org/abs/2105.05233}{Dhariwal P., Nichol A. Diffusion Models Beat GANs on Image Synthesis, 2021}
	\begin{block}{Unconditional Generation}
		\vspace{-0.3cm}
		\[
			\bx_{t - 1} = \frac{1}{\sqrt{1 - \beta_t}} \cdot \bx_t + \frac{\beta_t}{\sqrt{1 - \beta_t}} \cdot {\color{teal} \nabla_{\bx_t} \log \pt(\bx_t)} +  \sigma_t \cdot \bepsilon
		\]
		\vspace{-0.5cm}
	\end{block}
	\begin{block}{Conditional Generation}
		\vspace{-0.5cm}
		\[
			\bx_{t - 1} =  \frac{1}{\sqrt{1 - \beta_t}}\cdot \bx_t +  \frac{\beta_t}{\sqrt{1 - \beta_t}}  \cdot  \nabla_{\bx_t} \log \pt(\bx_t | {\color{olive}\by}) +  \sigma_t \cdot \bepsilon
		\]
		\vspace{-0.5cm}
	\end{block}
	\begin{block}{Conditional Distribution}
		\vspace{-0.5cm}
		\begin{align*}
			{\color{olive}\nabla_{\bx_t} \log \pt(\bx_t | \by)} &= \nabla_{\bx_t} \log p(\by | \bx_t) + {\color{violet}\nabla_{\bx_t} \log \pt(\bx_t)}\\
			&= {\color{teal}\nabla_{\bx_t} \log p(\by | \bx_t)} {\color{violet}- \frac{\bepsilon_{\btheta, t}(\bx_t)}{\sqrt{1 - \bar{\alpha}_t}}}
		\end{align*}
		\vspace{-0.5cm}
	\end{block}
	Here, $p(\by | \bx_t)$ denotes a classifier operating on noisy samples (which must be trained separately).
\end{frame}
%=======
\begin{frame}{Recap of Previous Lecture}
	\myfootnotewithlink{https://arxiv.org/abs/2105.05233}{Dhariwal P., Nichol A. Diffusion Models Beat GANs on Image Synthesis, 2021}
	\begin{block}{Classifier-Corrected Noise Prediction}
		\vspace{-0.3cm}
		\[
			\bepsilon_{\btheta, t}(\bx_t, \by) = \bepsilon_{\btheta, t}(\bx_t) - \sqrt{1 - \bar{\alpha}_t} \cdot \nabla_{\bx_t} \log p(\by | \bx_t)
		\]
		\vspace{-0.7cm}
	\end{block}
	\begin{block}{Guidance Scale}
		\vspace{-0.3cm}
		\[
			{\color{olive}\bepsilon_{\btheta, t}(\bx_t, \by)} = \bepsilon_{\btheta, t}(\bx_t) - {\color{teal}\gamma} \cdot \sqrt{1 - \bar{\alpha}_t} \cdot \nabla_{\bx_t} \log p(\by | \bx_t)
		\]
		\vspace{-0.5cm}
	\end{block}
	\begin{itemize}
		\item Train DDPM as usual.
		\item Train a separate classifier $p(\by | \bx_t)$ on noisy samples $\bx_t$.
	\end{itemize}
	\begin{block}{Guided Sampling}
		\vspace{-0.3cm}
		\[
			\bx_{t-1} = \frac{1}{\sqrt{\alpha_t}} \cdot \bx_t - \frac{1 - \alpha_t}{\sqrt{\alpha_t (1 - \bar{\alpha}_t)}} \cdot  {\color{olive}\bepsilon_{\btheta, t}(\bx_t, \by)} + \sigma_t \cdot \bepsilon
		\]
		\vspace{-0.3cm}
	\end{block}
	\textbf{Note:} The guidance scale $\gamma$ sharpens the distribution $p(\by | \bx_t)$.
\end{frame}
%=======
\begin{frame}{Recap of Previous Lecture}
	\myfootnotewithlink{https://arxiv.org/abs/2207.12598}{Ho J., Salimans T. Classifier-Free Diffusion Guidance, 2022}
	The previous method requires an additional classifier $p(\by | \bx_t)$ trained on noisy data. Let's try to avoid this requirement.
	\[
		{\color{teal}\nabla_{\bx_t} \log p(\by | \bx_t)} =  \nabla_{\bx_t} \log \pt(\bx_t| \by) -\nabla_{\bx_t} \log  \pt(\bx_t)
	\]
	\vspace{-0.4cm}
	\begin{multline*}
		\nabla_{\bx_t}^{\gamma} \log \pt(\bx_t | \by) = \nabla_{\bx_t} \log \pt(\bx_t) + \gamma \cdot \nabla_{\bx_t} {\color{violet}\log p(\by | \bx_t)} = \\
		=  (1 - \gamma) \cdot  \nabla_{\bx_t} \log \pt(\bx_t) + \gamma \cdot  \nabla_{\bx_t} \log \pt(\bx_t| \by)
	\end{multline*}
	\vspace{-0.4cm}
	\begin{block}{Classifier-Free-Corrected Noise Prediction}
		\vspace{-0.3cm}
		\[
			\hat{\bepsilon}_{\btheta, t}(\bx_t, \by) = \gamma \cdot \bepsilon_{\btheta, t}(\bx_t, \by) + (1 - \gamma) \cdot \bepsilon_{\btheta, t}(\bx_t)
		\]
		\vspace{-0.5cm}
	\end{block}
	\begin{itemize}
		\item Train one model $\bepsilon_{\btheta, t}(\bx_t, \by)$ on \textbf{supervised} data, alternating between true conditioning $\by$ and empty conditioning $\by = \emptyset$.
		\item During inference, apply this model twice.
	\end{itemize}
	
\end{frame}
%=======
\begin{frame}{Recap of Previous Lecture}
	\myfootnotewithlink{https://arxiv.org/abs/1806.07366}{Chen R. T. Q. et al. Neural Ordinary Differential Equations, 2018}   
	\begin{block}{Continuous-Time Dynamics (Neural ODE)}
		\vspace{-0.5cm}
		\begin{align*}
			\frac{d \bx(t)}{dt} &= \bff_{\btheta}(\bx(t), t);\quad \text{where } \bx(t_0) = \bx_0. \\
			\bpsi_t(\bx_0) &= \int^{t_1}_{t_0} \bff_{\btheta}(\bx(t), t) d t  + \bx_0 \approx {\color{teal}\texttt{ODESolve}_f(\bx_0, \btheta, t_0, t_1)}.
		\end{align*}
		\vspace{-0.5cm}
	\end{block}
	\begin{itemize}
		\item $\bff_{\btheta}: \bbR^m \times [t_0, t_1] \rightarrow \bbR^m$ is a vector field.
		\item $\bpsi: \bbR^m \times [t_0, t_1] \rightarrow \bbR^m$ is a flow (the solution of ODE):
	\end{itemize}
	\begin{block}{Euler Update Step (ODESolve)}
		\vspace{-0.3cm}
		\[
  			\bx(t + h) = \bx(t) + h \cdot \bff_{\btheta}(\bx(t), t)
		\]
	\end{block}
	More advanced numerical methods (such as Runge-Kutta) are often used instead of unstable Euler update step.
\end{frame}
%=======
\begin{frame}{Recap of Previous Lecture}
	\myfootnotewithlink{https://arxiv.org/abs/1810.01367}{Grathwohl W. et al. FFJORD: Free-form Continuous Dynamics for Scalable Reversible Generative Models, 2018}  
	\vspace{-0.3cm}
	\[
 		\frac{d \bx(t)}{dt} = \bff_{\btheta}(\bx(t), t);\quad \bx(t_0) = \bx_0
	\]
	\vspace{-0.3cm}
	\begin{itemize}
		\item Suppose $\bx(0)$ is a random variable with density~$p_0(\bx)$. Then, $\bx(t)$ is a random variable with density~$p_t(\bx)$.
		\item $p_t(\bx) = p(\bx, t)$ describes the \textbf{probability path} between $p_0(\bx)$ and $p_1(\bx)$.
	\end{itemize}
	\vspace{-0.3cm}
	\begin{figure}
		\centering
		\includegraphics[width=0.7\linewidth]{figs/cnf_flow.png}
	\end{figure}
\end{frame}
%=======
\begin{frame}{Recap of Previous Lecture}
	\myfootnotewithlink{https://arxiv.org/abs/1806.07366}{Chen R. T. Q. et al. Neural Ordinary Differential Equations, 2018}   
	\begin{block}{Theorem (Picard)}
		If $\bff$ is uniformly Lipschitz continuous in $\bx$ and continuous in $t$, then the ODE has a \textbf{unique} solution.
	\end{block}
	This means the ODE can be \textbf{uniquely inverted}:
	\begin{align*}
		\bx(1) &= \bx(0) + \int_{0}^{1} \bff_{\btheta}(\bx(t), t) dt \\
		\bx(0) &= \bx(1) + \int_{1}^{0} \bff_{\btheta}(\bx(t), t) dt
	\end{align*}
	\textbf{Note:} Unlike discrete-time NF, $\bff$ need not be invertible (uniqueness ensures bijectivity).
	
	How can we compute $p_t(\bx)$ for any $t$?
\end{frame}
%=======
\begin{frame}{Outline}
	\tableofcontents
\end{frame}
%=======
\section{Continuity Equation for NF Log-Likelihood}
%=======
\begin{frame}{Continuous-Time NF}
	\myfootnotewithlink{https://arxiv.org/abs/1806.07366}{Chen R. T. Q. et al. Neural Ordinary Differential Equations, 2018}   
	\begin{block}{Theorem (Continuity Equation)}
		If $\bff$ is uniformly Lipschitz continuous in $\bx$ and continuous in $t$, then
		\[
			\frac{d \log p_t(\bx(t))}{d t} = - \tr \left( \frac{\partial \bff(\bx(t), t)}{\partial \bx(t)} \right)
		\]
		\vspace{-0.3cm}
	\end{block}
	\eqpause
	This result states: given $\bx_0 = \bx(0)$, the solution to the continuity equation gives the density $p_1(\bx(1))$.
	\begin{block}{Solution of the Continuity Equation}
		\vspace{-0.3cm}
		\[
			\log p_1(\bx(1)) = \log p_0(\bx(0)) - {\color{teal}\int_{0}^{1} \tr  \left( \frac{\partial \bff(\bx(t), t)}{\partial \bx(t)} \right) dt}.
		\]
	\end{block}
	\eqpause
	\begin{itemize}
		\item This provides the density along the trajectory (not the total probability path).
		\item However, {\color{teal}the latter term} is difficult to estimate efficiently.
	\end{itemize}
\end{frame}
%=======
\section{SDE Basics}
%=======
\begin{frame}{Stochastic Differential Equation (SDE)}
	\myfootnotewithlink{https://arxiv.org/abs/2506.02070}{Holderrieth P., Erives E. An Introduction to Flow Matching and Diffusion Models, 2025}
	\begin{block}{Wiener Process}
		$\bw(t)$ is the standard Wiener process (Brownian motion), defined by:
		\vspace{-0.3cm}
		\begin{minipage}{0.5\columnwidth}
			\begin{enumerate}
				\item $\bw(0) = 0$ (almost surely);
				\item $\bw(t)$ has independent increments;
				\item $\bw(t)$ trajectories are continuous;
				\item $\bw(t) - \bw(s) \sim \cN(0, (t-s) \bI)$ for $t>s$;
			\end{enumerate}
		\end{minipage}%
		\begin{minipage}{0.45\columnwidth}
			\centering
			\includegraphics[width=\linewidth]{figs/brownian_motion}
		\end{minipage}
	\end{block}
	\eqpause
	$d\bw = \bw(t+dt) - \bw(t) = \cN(0, \bI \cdot dt ) = \bepsilon \cdot \sqrt{dt}$, where $\bepsilon \sim \cN(0, \bI)$.
\end{frame}
%=======
\begin{frame}{Stochastic Differential Equation (SDE)}
	\[
 		\frac{d \bx}{dt} = \bff(\bx, t) \Rightarrow d\bx = \bff(\bx, t) dt
	\]
	\eqpause
	Let's define a stochastic process $\bx(t)$ with initial condition $\bx(0) \sim p_0(\bx) = \pd(\bx)$:
	\[
		d\bx = \bff(\bx, t) dt + {\color{violet}g(t) d \bw}
	\]
	\eqpause
	\vspace{-0.5cm}
	\begin{itemize}
		 \item $\bff(\bx, t): \bbR^m \times [0, 1] \rightarrow \bbR^m$ is the \textbf{drift} term (vector field).
		 \item $g(t): \bbR \rightarrow \bbR$ is the \textbf{diffusion} term (if $g(t)=0$, we recover the standard ODE).
		 \item $\bw(t)$ is the standard Wiener process ($d\bw = \bepsilon \cdot \sqrt{dt}$).
		 \item We do not have the flow $\bpsi_t(\bx_0)$ notion anymore, since trajectories are stochastic
	\end{itemize}
\end{frame}
%=======
\begin{frame}{Stochastic Differential Equation (SDE)}
	\[
		d\bx = \bff(\bx, t) dt + g(t) d \bw
	\]
	\begin{block}{Theorem}
		If $\bff$ is continuously differentiable with a bounded derivative in $\bx$ and continuous in $t$ and $g(t)$ is continuous then the SDE has the solution given by unique proces $\bx(t)$.
	\end{block}

	\begin{itemize}
		\item Unlike ODEs, the initial condition $\bx(0)$ doesn't uniquely determine the trajectory.
		\item There are two sources of randomness: 
		\begin{itemize}
			\item the initial distribution $p_0(\bx)$;
			\item the Wiener process $\bw(t)$.
		\end{itemize}
	\end{itemize}
\end{frame}
%=======
\begin{frame}{Stochastic Differential Equation (SDE)}
	\[
		d\bx = \bff(\bx, t) dt + g(t) d \bw
	\]
	\vspace{-0.3cm}
	\eqpause
	\begin{block}{Discretizing the SDE (Euler Update Step) – \texttt{SDESolve}}
		\vspace{-0.3cm}
		\[
			\bx(t + dt) = \bx(t) + \bff(\bx(t), t) \cdot dt + g(t) \cdot \bepsilon \cdot \sqrt{dt}
		\]
		If $dt=1$, then
		\vspace{-0.3cm}
		\[
			\bx_{t + 1} = \bx_t + \bff(\bx_t, t) + g(t) \cdot \bepsilon
		\]
		\vspace{-0.7cm}
	\end{block}
	\eqpause
	\begin{itemize}
		\item At any time $t$, the process has density $p_t(\bx) = p(\bx, t)$.
		\item $p: \bbR^m \times [0,1] \rightarrow \bbR_+$ specifies a \textbf{probability path} from $p_0(\bx)$ to $p_1(\bx)$.
		\item How can we obtain the probability path $p_t(\bx)$ for $\bx(t)$?
	\end{itemize}
\end{frame}
%=======
\begin{frame}{Stochastic Differential Equation (SDE)}
	\vspace{-0.4cm}
	\[
		d\bx = \bff(\bx, t) dt + g(t) d \bw,\quad d \bw = \bepsilon \cdot \sqrt{dt},\quad \bepsilon \sim \cN(0, \bI).
	\]
	\vspace{-0.4cm}
	\begin{block}{Theorem (Kolmogorov-Fokker-Planck)}
		The evolution of $p_t(\bx)$ is governed by
		\vspace{-0.2cm}
		\[
			\frac{\partial p_t(\bx)}{\partial t} = - \diver\left(\bff(\bx, t) p_t(\bx)\right) + \frac{1}{2}g^2(t) \Delta_{\bx}p_t(\bx)
		\]
		\eqpause
		Here,
		\[
			\diver (\bv) = \sum_{i=1}^m \frac{\partial v_i(\bx)}{\partial x_i} = \tr\left( \frac{\partial \bv(\bx)}{\partial \bx}  \right)
		\]
		\[
			\Delta_{\bx}p_t(\bx) = \sum_{i=1}^m \frac{\partial^2 p_t(\bx)}{\partial x_i^2} = \tr\left( \frac{\partial^2 p_t(\bx)}{\partial \bx^2}  \right)
		\]
		\eqpause
		\[
			\frac{\partial p_t(\bx)}{\partial t} = \tr\left( - \frac{\partial}{\partial \bx} \bigl[ \bff(\bx, t) p_t(\bx)\bigr] + \frac{1}{2} g^2(t) \frac{\partial^2 p_t(\bx)}{\partial \bx^2} \right)
		\]
	\end{block}
\end{frame}
%=======
\begin{frame}{Stochastic Differential Equation (SDE)}
	\begin{block}{Theorem (Kolmogorov-Fokker-Planck)}
		\vspace{-0.2cm}
		\[
			\frac{\partial p_t(\bx)}{\partial t} = \tr\left( - \frac{\partial}{\partial \bx} \bigl[ \bff(\bx, t) p_t(\bx)\bigr] + \frac{1}{2} g^2(t) \frac{\partial^2 p_t(\bx)}{\partial \bx^2}\right)
		\]
		\vspace{-0.2cm}
	\end{block}
	\begin{itemize}
		\item The KFP theorem is necessary and sufficient condition (it is uniquely defines $p_t(\bx)$).
		\item This generalizes the continuity equation for continuous-time NF:
		\[
			\frac{d \log p_t(\bx(t))}{d t} = - \tr \left( \frac{\partial \bff(\bx, t)}{\partial \bx} \right).
		\]
	\end{itemize}
	\eqpause
	\vspace{-0.3cm}
	\begin{block}{Special Case: constant density $p_t(\bx)$}
		Let find the SDE for which $p_t(\bx) = \text{const}$ (i.e., if $\bx(0) \sim p_0(\bx)$, then $\bx(t) \sim p_0(\bx)$.).
		\eqpause
		\[
			\frac{\partial p_t(\bx)}{\partial t} = 0 \quad \Leftrightarrow \quad \tr\left( - \frac{\partial}{\partial \bx} \bigl[ \bff(\bx, t) p_t(\bx)\bigr] + \frac{1}{2} g^2(t) \frac{\partial^2 p_t(\bx)}{\partial \bx^2}\right) = 0
		\]
	\end{block}
\end{frame}
%=======
\begin{frame}{Langevin SDE (Special Case)}
	\[
		\frac{\partial p_t(\bx)}{\partial t} = 0 \quad \Leftrightarrow \quad \tr\left( - \frac{\partial}{\partial \bx} \bigl[ \bff(\bx, t) p_t(\bx)\bigr] + \frac{1}{2} g^2(t) \frac{\partial^2 p_t(\bx)}{\partial \bx^2}\right) = 0
	\]
	\vspace{-0.3cm}
	\eqpause
	\[
		\frac{\partial}{\partial \bx} \bigl[ \bff(\bx, t) p_t(\bx)\bigr] = \frac{1}{2} g^2(t) \frac{\partial^2 p_t(\bx)}{\partial \bx^2}
	\]
	\eqpause
	\[
		\bff(\bx, t) p_t(\bx) = \frac{1}{2} g^2(t) \frac{\partial p_t(\bx)}{\partial \bx}
	\]
	\eqpause
	\[
		\bff(\bx, t) = \frac{1}{2} g^2(t) \frac{1}{p_t(\bx)}\frac{\partial p_t(\bx)}{\partial \bx} = \frac{1}{2} g^2(t) \frac{\partial}{\partial \bx} \log p_t(\bx)
	\]
	\eqpause
	Let ${\color{olive}g(t)=1}$, then ${\color{violet}\bff(\bx, t) = \frac{1}{2} \frac{\partial}{\partial \bx} \log p_t(\bx)}$.
	\[
		d \bx = {\color{violet}\frac{1}{2} \frac{\partial}{\partial \bx} \log p_t(\bx) d t} + {\color{olive}1} \cdot d \bw
	\]
\end{frame}
%=======
\begin{frame}{Langevin SDE (Special Case)}
	Let find the SDE for which $p_t(\bx) = \text{const}$ (i.e., if $\bx(0) \sim p_0(\bx)$, then $\bx(t) \sim p_0(\bx)$.).
	\[
		d \bx = {\color{violet}\frac{1}{2} \frac{\partial}{\partial \bx} \log p_t(\bx) d t} + {\color{olive}1} \cdot d \bw
	\]
	\eqpause
	\begin{block}{Discretized Langevin SDE}
		\vspace{-0.3cm}
		\[
			\bx_{t + 1} - \bx_t = \frac{\eta}{2} \cdot \frac{\partial}{\partial \bx} \log p_t(\bx) + \sqrt{\eta} \cdot \bepsilon, \quad \eta \approx dt.
		\]
		\vspace{-0.4cm}
	\end{block}
	\begin{block}{Langevin Dynamic}
		\vspace{-0.3cm}
		\[
			\bx_{t + 1} = \bx_t + \frac{\eta}{2} \cdot \nabla_{\bx} \log \pt(\bx) + \sqrt{\eta} \cdot \bepsilon, \quad \eta \approx dt.
		\]
		\vspace{-0.3cm}
	\end{block}
	We (partially) explained, why Langevin dynamics is working.
\end{frame}
%=======
\section{Probability Flow ODE}
%=======
\begin{frame}{Probability Flow ODE}
	\myfootnotewithlink{https://arxiv.org/abs/2011.13456}{Song Y., et al. Score-Based Generative Modeling through Stochastic Differential Equations, 2020}
	\begin{block}{ODE and Continuity Equation}
		\vspace{-0.3cm}
		\[
			d\bx = \bff(\bx, t) dt
		\]
		\[
			\frac{d \log p_t(\bx(t))}{d t} = - \tr \left( \frac{\partial \bff_{\btheta}(\bx, t)}{\partial \bx} \right) 
			\quad  \Leftrightarrow  \quad 
 			\frac{\partial p_t(\bx)}{\partial t} = - \diver(\bff(\bx, t) p_t(\bx))
 		\]
		The only source of randomness is the initial distribution $p_0(\bx)$.
	\end{block}
	\eqpause
	\begin{block}{SDE and KFP Equation}
		\vspace{-0.3cm}
		\[
			d\bx = \bff(\bx, t) dt + g(t) d \bw
		\]
 		\[
 			\frac{\partial p_t(\bx)}{\partial t} = - \diver(\bff(\bx, t) p_t(\bx)) + \frac{1}{2}g^2(t) \Delta_{\bx}p_t(\bx)
 		\]
		Now there are two sources of randomness: the initial distribution $p_0(\bx)$ and the Wiener process $\bw(t)$.
	\end{block}
\end{frame}
%=======
\begin{frame}{Probability Flow ODE}
	\myfootnotewithlink{https://arxiv.org/abs/2011.13456}{Song Y., et al. Score-Based Generative Modeling through Stochastic Differential Equations, 2020}
	\begin{block}{Theorem}
		Suppose the SDE $d\bx = \bff(\bx, t) dt + g(t) d \bw$ induces the probability path $p_t(\bx)$.
		Then, there exists an ODE with the same probability path $p_t(\bx)$, given by
		\vspace{-0.3cm}
		\[
			d\bx = \left(\bff(\bx, t) -\frac{1}{2} g^2(t) \frac{\partial}{\partial \bx} \log p_t(\bx) \right) dt
		\]
		\vspace{-0.7cm}
	\end{block}
	\eqpause
	\begin{figure}
		\includegraphics[width=0.75\linewidth]{figs/probability_flow}
	\end{figure}
\end{frame}
%=======
\begin{frame}{Probability Flow ODE}
	\myfootnotewithlink{https://arxiv.org/abs/2011.13456}{Song Y., et al. Score-Based Generative Modeling through Stochastic Differential Equations, 2020}
	\vspace{-0.3cm}
	\begin{block}{Theorem}
		Suppose the SDE $d\bx = \bff(\bx, t) dt + g(t) d \bw$ induces the probability path $p_t(\bx)$.
		Then, there exists an ODE with the same probability path $p_t(\bx)$, given by
		\vspace{-0.3cm}
		\[
			d\bx = \left(\bff(\bx, t) -\frac{1}{2} g^2(t) \frac{\partial}{\partial \bx} \log p_t(\bx) \right) dt
		\]
		\vspace{-0.7cm}
	\end{block}
	\begin{block}{Proof}
 		\vspace{-0.7cm}
 		{\small
 		\begin{multline*}
 			\frac{\partial p_t(\bx)}{\partial t} = \tr\left( - \frac{\partial}{\partial \bx} \bigl[ \bff(\bx, t) p_t(\bx)\bigr] + \frac{1}{2} g^2(t) \frac{\partial^2 p_t(\bx)}{\partial \bx^2} \right) 
			\nextonslide{= \\ = \tr\left( - \frac{\partial}{\partial \bx} \left[ \bff(\bx, t) p_t(\bx) - \frac{1}{2} g^2(t) {\color{violet}\frac{\partial p_t(\bx)}{\partial \bx}} \right]  \right)}
			\nextonslide{ = \\ =  \tr\left( - \frac{\partial}{\partial \bx} \left[ \bff(\bx, t) p_t(\bx) - \frac{1}{2} g^2(t) {\color{violet}p_t(\bx) \frac{\partial}{\partial \bx} \log p_t(\bx)} \right]  \right)}
			\nextonslide{= \\ =  \tr\left( - \frac{\partial}{\partial \bx} \left[ \left( {\color{teal}\bff(\bx, t) - \frac{1}{2} g^2(t) \frac{\partial}{\partial \bx} \log p_t(\bx)}\right) p_t(\bx) \right]  \right)}
 		\end{multline*}
 		}
 	\end{block}
\end{frame}
%=======
\begin{frame}{Probability Flow ODE}
	\myfootnotewithlink{https://arxiv.org/abs/2011.13456}{Song Y., et al. Score-Based Generative Modeling through Stochastic Differential Equations, 2020}
	\begin{block}{Proof (Continued)}
 		\vspace{-0.7cm}
 		\begin{multline*}
 			\frac{\partial p_t(\bx)}{\partial t} =  \tr\left( - \frac{\partial}{\partial \bx} \left[ \left( {\color{teal}\bff(\bx, t) - \frac{1}{2} g^2(t) \frac{\partial}{\partial \bx}\log p_t(\bx)}\right) p_t(\bx) \right]  \right) 
			\nextonslide{=\\ =  \tr\left( - \frac{\partial}{\partial \bx} \left[ {\color{teal}\tilde{\bff}(\bx, t)} p_t(\bx) \right]  \right) = -  \diver\left(\tilde{\bff}(\bx, t) p_t(\bx)\right)}
 		\end{multline*}
		\eqpause
		\[
			\tilde{\bff}(\bx, t) = \bff(\bx, t) -\frac{1}{2} g^2(t) \frac{\partial}{\partial \bx} \log p_t(\bx); \quad \tilde{g}(t) = 0
		\]
	 	\[
	 		d \bx = \tilde{\bff}(\bx, t) dt + 0 \cdot d \bw = \left(\bff(\bx, t) -\frac{1}{2} g^2(t) \frac{\partial}{\partial \bx} \log p_t(\bx) \right) dt
	 	\]
		\[
			\frac{\partial p_t(\bx)}{\partial t} = - \diver(\tilde{\bff}(\bx, t) p_t(\bx)) + \frac{1}{2}\tilde{g}^2(t) \Delta_{\bx}p_t(\bx)
		\]
 	\end{block}
\end{frame}
%=======
\begin{frame}{Probability Flow ODE}
	\myfootnotewithlink{https://arxiv.org/abs/2011.13456}{Song Y., et al. Score-Based Generative Modeling through Stochastic Differential Equations, 2020}
	\vspace{-0.3cm}
	\begin{align*}
		d\bx &= \bff(\bx, t) dt + g(t) d \bw \;\; - \text{SDE} \\
		d\bx &= \left(\bff(\bx, t) -\frac{1}{2} g^2(t) \frac{\partial}{\partial \bx} \log p_t(\bx) \right) dt  \;\; - \text{Probability Flow ODE}
	\end{align*}
	\eqpause
	\vspace{-0.5cm}
	\begin{itemize}
		\item The term $\bs(\bx, t) = \frac{\partial}{\partial \bx} \log p_t(\bx)$ is the score function in continuous time.
		\eqpause
		\item The ODE produces more stable trajectories.
	\end{itemize}
	\vspace{-0.3cm}
	\begin{figure}
		\includegraphics[width=0.75\linewidth]{figs/probability_flow}
	\end{figure}
\end{frame}
%=======
\section{Reverse SDE}
%=======
\begin{frame}{Reverse SDE}
	\myfootnotewithlink{https://arxiv.org/abs/2011.13456}{Song Y., et al. Score-Based Generative Modeling through Stochastic Differential Equations, 2020}
	\vspace{-0.3cm}
	\[
		d\bx = \bff(\bx, t) dt,\qquad \bx(t + dt) = \bx(t) + \bff(\bx, t) dt
	\]
	Here $dt$ can be $>0$ or $<0$.
	\eqpause
	\begin{block}{Reverse ODE}
		Let $\tau = 1 - t$ ($d\tau = -dt$).
		\vspace{-0.3cm}
		\[
			d\bx = - \bff(\bx, 1 - \tau) d\tau
		\]
	\end{block}
	\eqpause
	\vspace{-0.5cm}
	\begin{itemize}
		\item How do we reverse the SDE $d\bx = \bff(\bx, t) dt + g(t) d \bw$? 
		\item The Wiener process introduces randomness that must be reversed.
	\end{itemize}
	\eqpause
	\vspace{-0.3cm}
	\begin{block}{Theorem}
		There exists a reverse SDE for $d\bx = \bff(\bx, t) dt + g(t) d \bw$, given by:
		\vspace{-0.3cm}
		\[
			d\bx = \left(\bff(\bx, t) {\color{violet}- g^2(t) \frac{\partial}{\partial \bx}\log p_t(\bx)}\right) dt + g(t) d \bw
		\]
		\vspace{-0.5cm}\\
		where $dt<0$.
	\end{block}
\end{frame}
%=======
\begin{frame}{Reverse SDE}
	\myfootnotewithlink{https://arxiv.org/abs/2011.13456}{Song Y., et al. Score-Based Generative Modeling through Stochastic Differential Equations, 2020}
	\begin{block}{Theorem}
		There exists a reverse SDE for $d\bx = \bff(\bx, t) dt + g(t) d \bw$, given by:
		\vspace{-0.3cm}
		\[
			d\bx = \left(\bff(\bx, t) {\color{violet}- g^2(t) \frac{\partial}{\partial \bx}\log p_t(\bx)}\right) dt + g(t) d \bw
		\]
		\vspace{-0.5cm}\\
		where $dt<0$.
	\end{block}
	\eqpause
	\textbf{Note:} Again, the score function appears: $\bs(\bx, t) = \frac{\partial}{\partial \bx} \log p_t(\bx)$.
	\begin{block}{Proof Sketch}
		\begin{itemize}
			\item Convert the initial SDE to a probability flow ODE.
			\item Reverse the probability flow ODE.
			\item Convert the reversed probability flow ODE back to an SDE.
		\end{itemize}
	\end{block}
\end{frame}
%=======
\begin{frame}{Reverse SDE}
	\myfootnotewithlink{https://arxiv.org/abs/2011.13456}{Song Y., et al. Score-Based Generative Modeling through Stochastic Differential Equations, 2020}	
	\begin{block}{Proof}
		\begin{itemize}
			\item Convert the initial SDE to a probability flow ODE:
			\vspace{-0.1cm}
			{\footnotesize
			\begin{align*}
				d\bx &= \bff(\bx, t) dt + g(t) d \bw \\
				d\bx &= \left(\bff(\bx, t) -\frac{1}{2} g^2(t) \frac{\partial}{\partial \bx} \log p_t(\bx) \right) dt
			\end{align*}
			}
			\vspace{-0.5cm}
			\eqpause
			\item Reverse the probability flow ODE:
			\vspace{-0.1cm}
			{\footnotesize
			\begin{align*}
				d\bx &= \left(\bff(\bx, t) -\frac{1}{2} g^2(t) \frac{\partial}{\partial \bx} \log p_t(\bx) \right) dt \\
				d\bx &= \left(-\bff(\bx, 1 - \tau) + \frac{1}{2} g^2(1 - \tau) \frac{\partial}{\partial \bx} \log p_{1 - \tau}(\bx) \right) d\tau
			\end{align*}
			}
			\vspace{-0.5cm}
			\eqpause
			\item Convert the reversed probability flow ODE back to an SDE:
			\vspace{-0.1cm}
			{\footnotesize
			\begin{align*}
				d\bx &= \left(-\bff(\bx, 1 - \tau) + \frac{1}{2} g^2(1 - \tau) \frac{\partial}{\partial \bx} \log p_{1 - \tau}(\bx) \right) d \tau \\
				d\bx &= \left(-\bff(\bx, 1 - \tau) + g^2(1 - \tau) \frac{\partial}{\partial \bx} \log p_{1 - \tau}(\bx) \right) d \tau + g(1-\tau)d\bw
			\end{align*}
			}
		\end{itemize}
	\end{block}
\end{frame}
%=======
\begin{frame}{Reverse SDE}
	\myfootnotewithlink{https://arxiv.org/abs/2011.13456}{Song Y., et al. Score-Based Generative Modeling through Stochastic Differential Equations, 2020}
	\begin{block}{Theorem}
		There exists a reverse SDE for $d\bx = \bff(\bx, t) dt + g(t) d \bw$, given by:
		\vspace{-0.3cm}
		\[
			d\bx = \left(\bff(\bx, t) {\color{violet}- g^2(t) \frac{\partial}{\partial \bx}\log p_t(\bx)}\right) dt + g(t) d \bw
		\]
		\vspace{-0.5cm}\\
		where $dt<0$.
	\end{block}
	\begin{block}{Proof (Continued)}
		\vspace{-0.7cm}
		\[
			d\bx = \left(-\bff(\bx, 1 - \tau) + g^2(1 - \tau) \frac{\partial}{\partial \bx} \log p_{1 - \tau}(\bx) \right)d\tau + g(1 - \tau) d \bw
		\]
		\eqpause
		\[
			d\bx = \left(\bff(\bx, t) - g^2(t) \frac{\partial}{\partial \bx}\log p_t(\bx)\right)dt + g(t) d\bw
		\]
		Here $d\tau > 0$ and $dt < 0$.
	\end{block}
\end{frame}
%=======
\begin{frame}{Reverse SDE}
	\myfootnotewithlink{https://arxiv.org/abs/2011.13456}{Song Y., et al. Score-Based Generative Modeling through Stochastic Differential Equations, 2020}
	\vspace{-0.3cm}
	\begin{align*}
		d\bx &= \bff(\bx, t) dt + g(t) d \bw \;\; - \text{SDE} \\
		d\bx &= \left(\bff(\bx, t) -\frac{1}{2} g^2(t) \frac{\partial}{\partial \bx} \log p_t(\bx) \right) dt \;\; - \text{Probability Flow ODE} \\
		d\bx &= \left(\bff(\bx, t) - g^2(t) \frac{\partial}{\partial \bx}\log p_t(\bx)\right) dt + g(t) d \bw \;\; - \text{Reverse SDE}
	\end{align*}
	\eqpause
	\vspace{-0.7cm}
	\begin{itemize}
		\item This framework allows us to transform one distribution into another via an SDE with a prescribed probability path $p_t(\bx)$.
		\item We can invert this process using the score function.
	\end{itemize}
	\vspace{-0.3cm}
	\begin{figure}
		\includegraphics[width=0.9\linewidth]{figs/sde}
	\end{figure}
\end{frame}
%=======
\begin{frame}{Summary}
	\begin{itemize}
		\item The continuity equation allows us to compute $\log p(\bx, t)$ at any time $t$.
		\vfill
		\item An SDE defines a stochastic process with drift and diffusion terms; ODEs are a special case of SDEs.
		\vfill
		\item The KFP equation describes the probability dynamics of an SDE.
		\vfill
		\item The Langevin SDE preserves a constant probability path.
		\vfill
		\item Every SDE admits a corresponding probability flow ODE following the same probability path.
		\vfill
		\item SDEs can be reversed using the score function.
	\end{itemize}
\end{frame}
\end{document}