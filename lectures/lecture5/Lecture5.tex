\documentclass{beamer}
\usepackage[utf8]{inputenc}
\usepackage{graphicx, epsfig}
\usepackage{amsmath,mathrsfs,amsfonts,amssymb}
\usepackage{floatflt}
\usepackage{epic,ecltree}
\usepackage{mathtext}
\usepackage{fancybox}
\usepackage{fancyhdr}
\usepackage{multirow}
\usepackage{enumerate}
\usepackage{epstopdf}
\usepackage{multicol}
\usepackage{algorithm}
\usepackage[noend]{algorithmic}
\usepackage{tikz}
\usepackage{blindtext}
\usepackage{multido}
\usetheme{default}%{Singapore}%{Warsaw}%{Warsaw}%{Darmstadt}
\usecolortheme{default}

\setbeamerfont{title}{size=\Huge}
\setbeamertemplate{footline}[frame number]{}

\setbeamertemplate{section in toc}[sections numbered]

\makeatletter
\newcommand\HUGE{\@setfontsize\Huge{35}{40}}
\makeatother    

\setbeamerfont{title}{size=\HUGE}
\beamertemplatenavigationsymbolsempty

\usetikzlibrary{arrows,shapes,positioning,shadows,trees}

\newcommand\myfootnote[1]{%
  \vspace{-0.5cm}%
  \tikz[remember picture,overlay]
  \draw (current page.south west) +(1in + \oddsidemargin,0.5em)
  node[anchor=south west,inner sep=0pt]{\parbox{\textwidth}{%
      \rlap{\rule{10em}{0.4pt}}\raggedright\scriptsize \textit{#1}}};}

\newcommand\myfootnotewithlink[2]{%
  \vspace{-0.5cm}%
  \tikz[remember picture,overlay]
  \draw (current page.south west) +(1in + \oddsidemargin,0.5em)
  node[anchor=south west,inner sep=0pt]{\parbox{\textwidth}{%
      \rlap{\rule{10em}{0.4pt}}\raggedright\scriptsize\href{#1}{\textit{#2}}}};}

\AtBeginSection[]
      {
      	\begin{frame}{Outline}
      		\tableofcontents[currentsection]
      	\end{frame}
      }
      \AtBeginSubsection[]{
      	\begin{frame}{Outline}
      		\tableofcontents[currentsection,currentsubsection]
      	\end{frame}
}

\newcounter{noscounter}
\newcounter{pcounter}
\newcommand{\nextonslide}[1]{%
  \stepcounter{noscounter}%
  \stepcounter{pcounter}%
  \onslide<\value{noscounter}->{#1}%
}
\newcommand{\resetonslide}{%
    \setcounter{noscounter}{1}%
    \setcounter{pcounter}{0}%
}

\newcommand{\eqpause}{%
  \multido{\i=1+1}{\value{pcounter}}{\pause}%
  \stepcounter{noscounter}%
  \setcounter{pcounter}{0}%
  \pause%
}
\addtobeamertemplate{frametitle}{\resetonslide}{}

\input{../utils/newcommands}
\input{../utils/title}
\createdgmtitle{5}
%--------------------------------------------------------------------------------
\begin{document}
%--------------------------------------------------------------------------------
\begin{frame}[noframenumbering,plain]
	\titlepage
	\resetonslide
\end{frame}
%=======
\begin{frame}{Recap of Previous Lecture}
	\myfootnotewithlink{https://arxiv.org/abs/1711.00937}{Oord A., Vinyals O., Kavukcuoglu K. Neural Discrete Representation Learning, 2017} 
	\vspace{-0.3cm}
	\begin{block}{Assumptions}
		\begin{itemize}
			\item Let $c \sim \Cat(\bpi)$, where 
			\vspace{-0.6cm}
			\[
				\bpi = (\pi_1, \dots, \pi_K), \quad \pi_k = P(c = k), \quad \sum_{k=1}^K \pi_k = 1.
			\]
			\vspace{-0.7cm}
			\item Suppose the VAE includes a discrete latent variable $c$ with prior $p(c) = \text{Uniform}\{1, \dots, K\}$.
		\end{itemize}
	\end{block}
	\vspace{-0.3cm}
	\begin{block}{ELBO}
		\vspace{-0.6cm}
		\[
			\cL_{\bphi, \btheta}(\bx)  = \bbE_{q_{\bphi}(c| \bx)} \log \pt(\bx| c) - {\color{olive} \KL(q_{\bphi}(c| \bx) \| p(c))} \rightarrow \max_{\bphi, \btheta}.
		\]
	\end{block}
	\vspace{-1.0cm}
	\[
		\KL(q_{\bphi}(c| \bx) \| p(c)) = - \Ent(q_{\bphi}(c| \bx)) + \log K. 
	\]		
	\vspace{-0.5cm}
	\begin{itemize}
		\item Our encoder must output the discrete distribution $q_{\bphi}(c| \bx)$.
		\item We'll require an analogue of the reparameterization trick for discrete $q_{\bphi}(c| \bx)$.
		\item Our decoder $\pt(\bx| c)$ must accept the discrete variable $c$ as input.
	\end{itemize}
\end{frame}
%=======
\begin{frame}{Outline}
	\tableofcontents
\end{frame}
%=======
\begin{frame}{Summary}
	\begin{itemize}
		\item
	\end{itemize}
\end{frame}
%=======
\end{document}