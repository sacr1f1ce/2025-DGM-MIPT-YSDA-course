\documentclass{beamer}
\usepackage[utf8]{inputenc}
\usepackage{graphicx, epsfig}
\usepackage{amsmath,mathrsfs,amsfonts,amssymb}
\usepackage{floatflt}
\usepackage{epic,ecltree}
\usepackage{mathtext}
\usepackage{fancybox}
\usepackage{fancyhdr}
\usepackage{multirow}
\usepackage{enumerate}
\usepackage{epstopdf}
\usepackage{multicol}
\usepackage{algorithm}
\usepackage[noend]{algorithmic}
\usepackage{tikz}
\usepackage{blindtext}
\usepackage{multido}
\usetheme{default}%{Singapore}%{Warsaw}%{Warsaw}%{Darmstadt}
\usecolortheme{default}

\setbeamerfont{title}{size=\Huge}
\setbeamertemplate{footline}[frame number]{}

\setbeamertemplate{section in toc}[sections numbered]

\makeatletter
\newcommand\HUGE{\@setfontsize\Huge{35}{40}}
\makeatother    

\setbeamerfont{title}{size=\HUGE}
\beamertemplatenavigationsymbolsempty

\usetikzlibrary{arrows,shapes,positioning,shadows,trees}

\newcommand\myfootnote[1]{%
  \vspace{-0.5cm}%
  \tikz[remember picture,overlay]
  \draw (current page.south west) +(1in + \oddsidemargin,0.5em)
  node[anchor=south west,inner sep=0pt]{\parbox{\textwidth}{%
      \rlap{\rule{10em}{0.4pt}}\raggedright\scriptsize \textit{#1}}};}

\newcommand\myfootnotewithlink[2]{%
  \vspace{-0.5cm}%
  \tikz[remember picture,overlay]
  \draw (current page.south west) +(1in + \oddsidemargin,0.5em)
  node[anchor=south west,inner sep=0pt]{\parbox{\textwidth}{%
      \rlap{\rule{10em}{0.4pt}}\raggedright\scriptsize\href{#1}{\textit{#2}}}};}

\AtBeginSection[]
      {
      	\begin{frame}{Outline}
      		\tableofcontents[currentsection]
      	\end{frame}
      }
      \AtBeginSubsection[]{
      	\begin{frame}{Outline}
      		\tableofcontents[currentsection,currentsubsection]
      	\end{frame}
}

\newcounter{noscounter}
\newcounter{pcounter}
\newcommand{\nextonslide}[1]{%
  \stepcounter{noscounter}%
  \stepcounter{pcounter}%
  \onslide<\value{noscounter}->{#1}%
}
\newcommand{\resetonslide}{%
    \setcounter{noscounter}{1}%
    \setcounter{pcounter}{0}%
}

\newcommand{\eqpause}{%
  \multido{\i=1+1}{\value{pcounter}}{\pause}%
  \stepcounter{noscounter}%
  \setcounter{pcounter}{0}%
  \pause%
}
\addtobeamertemplate{frametitle}{\resetonslide}{}

\input{../utils/newcommands}
\input{../utils/title}
\createdgmtitle{8}
%--------------------------------------------------------------------------------
\begin{document}
%--------------------------------------------------------------------------------
\begin{frame}[noframenumbering,plain]
\titlepage
\end{frame}
%=======
\begin{frame}{Recap of Previous Lecture}
    \myfootnotewithlink{https://arxiv.org/abs/1706.08500}{Heusel M. et al. GANs Trained by a Two Time-Scale Update Rule Converge to a Local Nash Equilibrium, 2017}
	\vspace{-0.3cm}
	\begin{block}{Frechet Inception Distance (FID)}
		For normal distributions $\pd(\bx_1) = \cN(\bmu_{\text{data}}, \bSigma_{\text{data}})$, $p(\bx_2) = \cN(\bmu_{\boldsymbol{\theta}}, \bSigma_{\boldsymbol{\theta}})$:
		\vspace{-0.3cm}
		\begin{multline*}
			\FID (\pd, \pt) =  W_2^2(\pd, \pt) = \inf_{\gamma \in \Gamma(\pd, \pt)} \bbE_{(\bx_1, \bx_2) \sim \gamma} \| \bx_1 - \bx_2 \|^2 \\
			= \| \bmu_{\text{data}} - \bmu_{\boldsymbol{\theta}}\|^2 + \text{tr} \left[ \bSigma_{\text{data}} + \bSigma_{\boldsymbol{\theta}} - 2 \left(\bSigma_{\text{data}}^{1/2} \bSigma_{\boldsymbol{\theta}} \bSigma_{\text{data}}^{1/2} \right)^{1/2} \right]
		\end{multline*}
		\vspace{-0.4cm}
	\end{block}
	\begin{itemize}
		\item Requires a large sample size for evaluation
		\item FID computation is relatively slow
		\item Results are highly dependent on the chosen pretrained classifier
		\item Relies on the normality assumption
	\end{itemize}
\end{frame}
%=======
\begin{frame}{Recap of Previous Lecture}
    \myfootnotewithlink{https://arxiv.org/abs/1904.06991}{Kynkäänniemi T. et al. Improved precision and recall metric for assessing generative models, 2019}
	\vspace{-0.2cm}
	\begin{itemize}
		\item $\cS_{\pi} = \{\bx_i\}_{i=1}^{n} \sim \pd(\bx)$: real samples
		\item $\cS_{p} = \{\bx_i\}_{i=1}^{n} \sim \pt(\bx)$: generated samples
	\end{itemize}
	Define a binary indicator function:
	\vspace{-0.2cm}
	\[
		\bbI(\bx, \cS) = 
		\begin{cases}
			1, & \text{if}~ \exists~ \bx' \in \cS: \| \bx  - \bx'\|_2 \leq \| \bx' - \text{NN}_k(\bx', \cS)\|_2; \\
			0, & \text{otherwise}
		\end{cases}
	\]
	\vspace{-0.3cm}
	\[
		\text{Precision} (\cS_{\pi}, \cS_{p}) = \frac{1}{n} \sum_{\bx \in \cS_{p}} \bbI(\bx, \cS_{\pi}),
		\quad
		\text{Recall} (\cS_{\pi}, \cS_{p}) = \frac{1}{n} \sum_{\bx \in \cS_{\pi}} \bbI(\bx, \cS_{p})
	\]
	\vspace{-0.7cm}
	\begin{figure}
		\includegraphics[width=0.75\linewidth]{figs/pr_k_nearest}
	\end{figure}
	\vspace{-0.3cm}
	The samples are embedded using a pretrained network, as in FID evaluation.
\end{frame}
%=======
\begin{frame}{Recap of Previous Lecture}
    \myfootnotewithlink{https://arxiv.org/abs/2103.00020}{Radford A. et al. Learning transferable visual models from natural language supervision, 2021} 
	\vspace{-0.2cm}
	\begin{minipage}{0.5\linewidth}
		\begin{block}{Unconditional Model}
			\begin{figure}
				\includegraphics[width=0.95\linewidth]{figs/uncond_model}
			\end{figure}
		\end{block}
	\end{minipage}%
	\begin{minipage}{0.5\linewidth}
		\vspace{0.2cm}
		\begin{block}{Conditional Model}
			\begin{figure}
				\includegraphics[width=0.95\linewidth]{figs/cond_model}
			\end{figure}
		\end{block}
	\end{minipage}
	We require metrics that evaluate not only the quality of generated images, but also their relevance to the prompt.
	\begin{block}{CLIP Score}
		\begin{figure}
			\includegraphics[width=0.5\linewidth]{figs/clip}
		\end{figure}
	\end{block}
\end{frame}
%=======
\begin{frame}{Recap of Previous Lecture}
    \myfootnotewithlink{https://ya.ru/ai/art}{YandexART 2.5, 2025} 
	\begin{itemize}
		\item No perfect automatic evaluation metric exists
		\item The most reliable assessment is via human evaluation
		\item It's important to evaluate a variety of model aspects
	\end{itemize}
	\begin{block}{Human Evaluation}
		\begin{figure}
			\includegraphics[width=1.0\linewidth]{figs/yaart_2.5}
		\end{figure}
	\end{block}
\end{frame}
%=======
\begin{frame}{Recap of Previous Lecture}
    \myfootnotewithlink{https://yang-song.github.io/blog/2021/score/}{Song Y. Generative Modeling by Estimating Gradients of the Data Distribution, blog post, 2021}
		\vspace{-0.3cm}
	\begin{block}{Langevin Dynamics}
		\vspace{-0.3cm}
		\[
			\bx_{l + 1} = \bx_l + \frac{\eta}{2} \cdot \nabla_{\bx_l} \log \pt(\bx_l) + \sqrt{\eta} \cdot \bepsilon_l, \quad \bepsilon_l \sim \cN(0, \bI)
		\]
		\vspace{-0.7cm}
	\end{block}
	\begin{block}{Fisher Divergence}
		\vspace{-0.3cm}
		\[
			D_F(\pi, p) = \frac{1}{2}\bbE_{\pi}\left\| \nabla_{\bx}\log \pt(\bx) - \nabla_\bx \log \pd(\bx) \right\|_2^2 \rightarrow \min_{\btheta}
		\]
		\vspace{-0.7cm}
	\end{block}
	\begin{block}{Score Function}
		\vspace{-0.5cm}
		 \[
			 \bs_{\btheta}(\bx) = \nabla_{\bx}\log \pt(\bx)
		 \]
	 \vspace{-0.8cm}
	\end{block}
	\begin{figure}
		\centering
		\includegraphics[width=0.75\linewidth]{figs/smld}
	\end{figure}
\end{frame}
%=======
\begin{frame}{Outline}
	\tableofcontents
\end{frame}
%=======
\section{Score Matching}
%=======
\subsection{Denoising Score Matching}
%=======
\begin{frame}{Denoising Score Matching}
    \myfootnotewithlink{http://www.iro.umontreal.ca/~vincentp/Publications/smdae_techreport.pdf}{Vincent P. A Connection Between Score Matching and Denoising Autoencoders, 2010}

	Let us perturb the original data $\bx \sim \pd(\bx)$ by adding Gaussian noise:
	\vspace{-0.3cm}
	\[
		\bx_{\sigma} = \bx + \sigma \bepsilon, \quad \bepsilon \sim \cN(0, \bI), \quad q(\bx_{\sigma} | \bx) = \cN(\bx, \sigma^2 \bI)
	\]
	\vspace{-0.5cm}
	\[
		q(\bx_{\sigma}) = \int q(\bx_{\sigma} | \bx) \pd(\bx) d\bx
	\]
    \eqpause
	\vspace{-0.5cm} 
	\begin{block}{Assumption}
		The solution to
		\[
			\frac{1}{2} \bbE_{q(\bx_{\sigma})}\left\| \bs_{\btheta, \sigma}(\bx_{\sigma}) - \nabla_{\bx_{\sigma}} \log q(\bx_{\sigma}) \right\|_2^2 \rightarrow \min_{\btheta}
		\]
		\vspace{-0.3cm}
		satisfies $\bs_{\btheta, \sigma}(\bx_{\sigma}) \approx \bs_{\btheta, 0}(\bx_0) = \bs_{\btheta}(\bx)$ if $\sigma$ is sufficiently small.
	\end{block}
    \eqpause
	\begin{itemize}
		\item The score function of the corrupted data closely approximates that of the original data
		\item The score function $\bs_{\btheta, \sigma}(\bx_{\sigma})$ is parameterized by $\sigma$
		\item \textbf{Note:} Neither $q(\bx_{\sigma})$ nor $\pd(\bx)$ are tractable
	\end{itemize}
\end{frame}
%=======
\begin{frame}{Denoising Score Matching}
    \myfootnotewithlink{http://www.iro.umontreal.ca/~vincentp/Publications/smdae_techreport.pdf}{Vincent P. A Connection Between Score Matching and Denoising Autoencoders, 2010}
	\begin{block}{Theorem}
		If $\text{supp}(q(\bx_{\sigma})) = \bbR^m$, then
		\begin{multline*}
			\bbE_{q(\bx_{\sigma})}\left\| \bs_{\btheta, \sigma}(\bx_{\sigma}) - \nabla_{\bx_{\sigma}} \log q(\bx_{\sigma}) \right\|_2^2 = \\
			= \bbE_{\pd(\bx)} \bbE_{q(\bx_{\sigma} | \bx)}\left\| \bs_{\btheta, \sigma}(\bx_{\sigma}) - \nabla_{\bx_{\sigma}} \log q(\bx_{\sigma} | \bx) \right\|_2^2 + \text{const}(\btheta)
		\end{multline*}
		\vspace{-0.5cm}
	\end{block}
    \eqpause
	\begin{block}{Gradient of the Noise Kernel}
		\vspace{-0.3cm}
		\[
			\bx_{\sigma} = \bx + \sigma \bepsilon, \quad q(\bx_{\sigma} | \bx) = \cN(\bx, \sigma^2 \bI)
		\]
		\vspace{-0.3cm}
		\[
			\nabla_{\bx_{\sigma}} \log q(\bx_{\sigma} | \bx) = - \frac{\bx_{\sigma} - \bx}{\sigma^2}  = - \frac{\bepsilon}{\sigma}
		\]
		\vspace{-0.5cm}
	\end{block}
    \eqpause
	\begin{itemize}
		\item The right-hand side does not require evaluating $\nabla_{\bx_{\sigma}} \log q(\bx_{\sigma})$ or $\nabla_{\bx_{\sigma}} \log \pd(\bx_{\sigma})$
		\item The neural network $\bs_{\btheta, \sigma}(\bx_{\sigma})$ is trained to \textbf{denoise} the corrupted samples $\bx_{\sigma}$
	\end{itemize}
\end{frame}
%=======
\begin{frame}{Denoising Score Matching}
    \myfootnotewithlink{http://www.iro.umontreal.ca/~vincentp/Publications/smdae_techreport.pdf}{Vincent P. A Connection Between Score Matching and Denoising Autoencoders, 2010}
	\begin{block}{Theorem}
	\vspace{-0.5cm}
	\begin{multline*}
		\bbE_{q(\bx_{\sigma})}\underbrace{\left\| \bs_{\btheta, \sigma}(\bx_{\sigma}) - \nabla_{\bx_{\sigma}} \log q(\bx_{\sigma}) \right\|_2^2}_{h(\bx_{\sigma})} = \\
		= \bbE_{\pd(\bx)} \bbE_{q(\bx_{\sigma} | \bx)}\left\| \bs_{\btheta, \sigma}(\bx_{\sigma}) - \nabla_{\bx_{\sigma}} \log q(\bx_{\sigma} | \bx) \right\|_2^2 + \text{const}(\btheta)
	\end{multline*}
	\vspace{-0.5cm}
	\end{block}
    \eqpause
	\begin{block}{Proof}
		\vspace{-0.7cm}
		\begin{multline*}
			\bbE_{q(\bx_{\sigma})} h(\bx_{\sigma}) = \int {\color{violet}q(\bx_{\sigma})} h(\bx_{\sigma}) d\bx_{\sigma} 
			\nextonslide{= \\ = \int \left({\color{violet}\int q(\bx_{\sigma} | \bx) \pd(\bx) d\bx}\right) h(\bx_{\sigma}) d\bx_{\sigma} =  \bbE_{\pd(\bx)} \bbE_{q(\bx_{\sigma} | \bx)}  h(\bx_{\sigma})}
		\end{multline*}
        \eqpause
		\vspace{-0.7cm}
		{\small
		\begin{multline*}
			\bbE_{q(\bx_{\sigma})}\left\| \bs_{\btheta, \sigma}(\bx_{\sigma}) - \nabla_{\bx_{\sigma}} \log q(\bx_{\sigma}) \right\|_2^2 = \\ 
			= \bbE_{q(\bx_{\sigma})} \Bigl[\| \bs_{\btheta, \sigma}(\bx_{\sigma}) \|^2 + \underbrace{\| \nabla_{\bx_{\sigma}} \log q(\bx_{\sigma}) \|_2^2}_{\text{const}(\btheta)} - 2 {\color{teal}\bs_{\btheta, \sigma}^T(\bx_{\sigma}) \nabla_{\bx_{\sigma}} \log q(\bx_{\sigma})} \Bigr]
		\end{multline*}
		}
	\end{block}
\end{frame}
%=======
\begin{frame}{Denoising Score Matching}
    \myfootnotewithlink{http://www.iro.umontreal.ca/~vincentp/Publications/smdae_techreport.pdf}{Vincent P. A Connection Between Score Matching and Denoising Autoencoders, 2010}
	\begin{block}{Theorem}
		\vspace{-0.5cm}
		\begin{multline*}
			\bbE_{q(\bx_{\sigma})}\left\| \bs_{\btheta, \sigma}(\bx_{\sigma}) - \nabla_{\bx_{\sigma}} \log q(\bx_{\sigma}) \right\|_2^2 = \\
			= \bbE_{\pd(\bx)} \bbE_{q(\bx_{\sigma} | \bx)}\left\| \bs_{\btheta, \sigma}(\bx_{\sigma}) - \nabla_{\bx_{\sigma}} \log q(\bx_{\sigma} | \bx) \right\|_2^2 + \text{const}(\btheta)
		\end{multline*}
		\vspace{-0.5cm}
	\end{block}
    \eqpause
	\begin{block}{Proof (Continued)}
		\vspace{-0.7cm}
		{\small
		\begin{multline*}
			\bbE_{q(\bx_{\sigma})} \left[{\color{teal}\bs_{\btheta, \sigma}^T(\bx_{\sigma}) \nabla_{\bx_{\sigma}} \log q(\bx_{\sigma})} \right] = \int q(\bx_{\sigma}) \left[\bs_{\btheta, \sigma}^T(\bx_{\sigma}) \frac{\nabla_{\bx_{\sigma}} {\color{violet}q(\bx_{\sigma})}}{q(\bx_{\sigma})} \right] d\bx_{\sigma} 
			\nextonslide{= \\ = \int \left[\bs_{\btheta, \sigma}^T(\bx_{\sigma}) \nabla_{\bx_{\sigma}}\left({\color{violet}\int q(\bx_{\sigma} | \bx) \pd(\bx) d\bx}\right) \right] d\bx_{\sigma}}
			\nextonslide{ = \\ =  \int \int \pd(\bx) \left[\bs_{\btheta, \sigma}^T(\bx_{\sigma}) {\color{olive}\nabla_{\bx_{\sigma}}q(\bx_{\sigma} | \bx)} \right] d\bx_{\sigma} d\bx}
			\nextonslide{ = \\ = \int \int \pd(\bx) {\color{olive} q(\bx_{\sigma} | \bx)} \left[\bs_{\btheta, \sigma}^T(\bx_{\sigma}) {\color{olive}\nabla_{\bx_{\sigma}} \log q(\bx_{\sigma} | \bx)} \right] d\bx_{\sigma} d\bx}
			\nextonslide{ = \\ = \bbE_{\pd(\bx)} \bbE_{q(\bx_{\sigma} | \bx)} \left[{\color{teal}\bs_{\btheta, \sigma}^T(\bx_{\sigma}) \nabla_{\bx_{\sigma}} \log q(\bx_{\sigma} | \bx)} \right]}
		\end{multline*}
		}
	\end{block}
\end{frame}
%=======
\begin{frame}{Denoising Score Matching}
    \myfootnotewithlink{http://www.iro.umontreal.ca/~vincentp/Publications/smdae_techreport.pdf}{Vincent P. A Connection Between Score Matching and Denoising Autoencoders, 2010}
	\vspace{-0.3cm}
	\begin{block}{Theorem}
		\vspace{-0.7cm}
		\begin{multline*}
			\bbE_{q(\bx_{\sigma})}\underbrace{\left\| \bs_{\btheta, \sigma}(\bx_{\sigma}) - \nabla_{\bx_{\sigma}} \log q(\bx_{\sigma}) \right\|_2^2}_{h(\bx_{\sigma})} = \\
			= \bbE_{\pd(\bx)} \bbE_{q(\bx_{\sigma} | \bx)}\left\| \bs_{\btheta, \sigma}(\bx_{\sigma}) - \nabla_{\bx_{\sigma}} \log q(\bx_{\sigma} | \bx) \right\|_2^2 + \text{const}(\btheta)
		\end{multline*}
		\vspace{-0.9cm}
	\end{block}
    \eqpause
	\begin{block}{Proof (Continued)}
		\vspace{-0.3cm}
		\[
			\bbE_{q(\bx_{\sigma})} h(\bx_{\sigma})=  \bbE_{\pd(\bx)} \bbE_{q(\bx_{\sigma} | \bx)}  h(\bx_{\sigma})
		\]
        \eqpause
		{\small
		\[
			\bbE_{q(\bx_{\sigma})} \left[\bs_{\btheta, \sigma}^T(\bx_{\sigma}) \nabla_{\bx_{\sigma}} \log q(\bx_{\sigma}) \right] = \bbE_{\pd(\bx)} \bbE_{q(\bx_{\sigma} | \bx)} \left[\bs_{\btheta, \sigma}^T(\bx_{\sigma}) \nabla_{\bx_{\sigma}} \log q(\bx_{\sigma} | \bx) \right]
		\]
        \eqpause
		\vspace{-0.5cm}
		\begin{multline*}
			\bbE_{q(\bx_{\sigma})}\left\| \bs_{\btheta, \sigma}(\bx_{\sigma}) - \nabla_{\bx_{\sigma}} \log q(\bx_{\sigma}) \right\|_2^2 = \\ 
			= {\color{olive}\bbE_{\pd(\bx)} \bbE_{q(\bx_{\sigma} | \bx)}}\left[\| \bs_{\btheta, \sigma}(\bx_{\sigma}) \|^2 - 2 \bs_{\btheta, \sigma}^T(\bx_{\sigma}) {\color{teal}\nabla_{\bx_{\sigma}} \log q(\bx_{\sigma} | \bx)} \right] + \text{const}(\btheta)
			\nextonslide{= \\ = {\color{olive}\bbE_{\pd(\bx)} \bbE_{q(\bx_{\sigma} | \bx)}} \left\|\bs_{\btheta, \sigma}(\bx_{\sigma}) - {\color{teal}\nabla_{\bx_{\sigma}} \log q(\bx_{\sigma} | \bx)} \right\|_2^2 + \text{const}(\btheta)}
		\end{multline*}
		}
		\vspace{-0.8cm}
	\end{block}
\end{frame}
%=======
\begin{frame}{Denoising Score Matching}
    \myfootnotewithlink{https://yang-song.github.io/blog/2021/score/}{Song Y. Generative Modeling by Estimating Gradients of the Data Distribution, blog post, 2021}
	Original objective:
	\vspace{-0.2cm}
	\[
		\bbE_{\pd(\bx)}\left\| \bs_{\btheta}(\bx) - \nabla_\bx \log \pd(\bx) \right\|_2^2 \rightarrow \min_{\btheta}
	\]
    \eqpause
	\vspace{-0.5cm} \\
	Noisy objective:
	\vspace{-0.2cm}
	\[
		\bbE_{q(\bx_{\sigma})}\left\| \bs_{\btheta, \sigma}(\bx_\sigma) - \nabla_\bx \log q(\bx_{\sigma}) \right\|_2^2 \rightarrow \min_{\btheta}
	\]
    \eqpause
	\vspace{-0.5cm} \\
	This is equivalent to a denoising task:
	\vspace{-0.2cm}
	\[
		\bbE_{\pd(\bx)} \bbE_{q(\bx_{\sigma} | \bx)}\left\| \bs_{\btheta, \sigma}(\bx_{\sigma}) - \nabla_{\bx_{\sigma}} \log q(\bx_{\sigma} | \bx) \right\|_2^2 \rightarrow \min_{\btheta}
	\]
    \eqpause
	\vspace{-0.3cm}
	\[
		\bbE_{\pd(\bx)} \bbE_{\cN(0, \bI)}\left\| \bs_{\btheta, \sigma}(\bx + \sigma \bepsilon) + \frac{\bepsilon}{\sigma} \right\|_2^2 \rightarrow \min_{\btheta}
	\]
    \eqpause
	\vspace{-0.5cm}
	\begin{block}{Langevin Dynamics}
		\vspace{-0.3cm}
		\[
			\bx_{l + 1} = \bx_l + \frac{\eta}{2} \cdot \bs_{\btheta, \sigma}(\bx_l) + \sqrt{\eta} \cdot \bepsilon_l, \quad \bepsilon_l \sim \cN(0, \bI)
		\]
		\vspace{-0.7cm}
	\end{block}
\end{frame}
%=======
\subsection{Noise-Conditioned Score Network}
%=======
\begin{frame}{Denoising Score Matching}
    \myfootnotewithlink{https://yang-song.github.io/blog/2021/score/}{Song Y. Generative Modeling by Estimating Gradients of the Data Distribution, blog post, 2021}
	\vspace{-0.5cm}
	\[
		\bbE_{\pd(\bx)} \bbE_{\cN(0, \bI)}\left\| \bs_{\btheta, \sigma}(\bx + \sigma \bepsilon) + \frac{\bepsilon}{\sigma} \right\|_2^2 \rightarrow \min_{\btheta}
	\]
	\begin{minipage}{0.5\linewidth}
		\[
			\bx_{l + 1} = \bx_l + \frac{\eta}{2} \cdot \bs_{\btheta, \sigma}(\bx_l) + \sqrt{\eta} \cdot \bepsilon_l
		\]
		\vspace{-0.3cm}
		\begin{itemize}
			\item For \textbf{small} $\sigma$, $\bs_{\btheta, \sigma}(\bx)$ becomes inaccurate and Langevin dynamics fails to traverse modes
			\item For \textbf{large} $\sigma$, robustness in low-density regions is achieved, but the model learns a distribution that is overly corrupted
		\end{itemize}
	\end{minipage}%
	\begin{minipage}{0.5\linewidth}
		\begin{figure}
			\includegraphics[width=\linewidth]{figs/pitfalls}
		\end{figure}
		\vspace{-0.3cm}
		\begin{figure}
			\includegraphics[width=\linewidth]{figs/single_noise}
		\end{figure}
	\end{minipage}
\end{frame}
%=======
\begin{frame}{Noise-Conditioned Score Network (NCSN)}
    \myfootnotewithlink{https://arxiv.org/abs/1907.05600}{Song Y. et al. Generative Modeling by Estimating Gradients of the Data Distribution, 2019}
	\begin{itemize}
		\item Specify a sequence of noise levels: $\sigma_1 < \sigma_2 < \dots < \sigma_T$
		\item Perturb each data point with different noise levels: $\bx_t = \bx + \sigma_t \bepsilon$, so $\bx_t \sim q(\bx_t)$
		\item Choose $\sigma_1, \sigma_T$ such that:
		\[
			q(\bx_1) \approx \pd(\bx), \;\; q(\bx_T) \approx \cN(0, \sigma_T^2 \bI)
		\]
	\end{itemize}
    \eqpause
	\vspace{-0.3cm}
	\begin{figure}
		\includegraphics[width=0.6\linewidth]{figs/multi_scale}
	\end{figure}
	\begin{figure}
		\includegraphics[width=\linewidth]{figs/duoduo}
	\end{figure}
\end{frame}
%=======
\begin{frame}{Noise-Conditioned Score Network (NCSN)}
    \myfootnotewithlink{https://arxiv.org/abs/1907.05600}{Song Y. et al. Generative Modeling by Estimating Gradients of the Data Distribution, 2019}
	Train the denoising score function $\bs_{\btheta, \sigma_t}(\bx_t)$ for each noise level $\sigma_t$ using a unified weighted objective:
	\vspace{-0.2cm}
	\[
		\sum_{t=1}^T {\color{violet}\sigma_t^2} \bbE_{\pd(\bx)} \bbE_{q(\bx_t | \bx)}\left\| \bs_{\btheta, \sigma_t}(\bx_t) - \nabla_{\bx_t} \log q(\bx_t | \bx) \right\|_2^2 \rightarrow \min_{\btheta}
	\]
    \eqpause
	Here, $\nabla_{\bx_t} \log q(\bx_t | \bx) = - \frac{\bx_t - \bx}{\sigma_t^2} = - \frac{\bepsilon}{\sigma_t}$
    \eqpause
	\begin{block}{Training Procedure}
		\begin{enumerate}
			\item Sample $\bx_0 \sim \pd(\bx)$
			\item Sample $t \sim U\{1, T\}$ and $\bepsilon \sim \cN(0, \bI)$
			\item Construct noisy image $\bx_t = \bx_0 + \sigma_t \bepsilon$
			\item Evaluate loss $ \cL = \sigma_t^2 \left\| \bs_{\btheta, \sigma_t}(\bx_t) + \frac{\bepsilon}{\sigma_t} \right\|^2 $
		\end{enumerate}
		\vspace{-0.3cm}
	\end{block}
    \eqpause
	How do we sample from such a model?
\end{frame}
%=======
\begin{frame}{Noise-Conditioned Score Network (NCSN)}
    \myfootnotewithlink{https://arxiv.org/abs/2006.09011}{Song Y. et al. Improved Techniques for Training Score-Based Generative Models, 2020}
	\begin{block}{Sampling (Annealed Langevin Dynamics)}
		\begin{itemize}
			\item Sample initial point $\bx_0 \sim \cN(0, \sigma_T^2 \bI) \approx q(\bx_T)$
			\item At each noise level, apply $L$ steps of Langevin dynamics:
			\vspace{-0.2cm}
			\[
				\bx_l = \bx_{l-1} + \frac{\eta_t}{2} \bs_{\btheta, \sigma_t}(\bx_{l-1}) + \sqrt{\eta_t} \bepsilon_l,
			\] 
			\vspace{-0.5cm}
			\item Update $\bx_0 := \bx_L$ and reduce to the next lower $\sigma_t$
		\end{itemize}
	\end{block}
    \eqpause
	\begin{figure}
		\includegraphics[width=0.9\linewidth]{figs/ald}
	\end{figure}
\end{frame}
%=======
\section{Forward Gaussian Diffusion Process}
%=======
\begin{frame}{Forward Gaussian Diffusion Process}
    \myfootnotewithlink{http://proceedings.mlr.press/v37/sohl-dickstein15.pdf}{Sohl-Dickstein J. Deep Unsupervised Learning using Nonequilibrium Thermodynamics, 2015}

	Let $\bx_0 = \bx \sim \pd(\bx)$, $\beta_t \ll 1$. Define a Markov chain:
	\[
		\bx_t = \sqrt{1 - \beta_t} \bx_{t-1} + \sqrt{\beta_t} \bepsilon_t, \;\; \bepsilon_t \sim \cN(0, \bI)
	\]
	\vspace{-0.5cm}
    \eqpause
	\[
		q(\bx_t | \bx_{t-1}) = \cN(\sqrt{1 - \beta_t} \bx_{t-1}, \beta_t \bI)
	\]
    \eqpause
	\vspace{-0.5cm}
	\begin{block}{Langevin Dynamics}
		\vspace{-0.3cm}
		\[
			\bx_{l + 1} = \bx_l + \frac{ \color{violet} \eta}{2} \cdot {\color{teal}\nabla_{\bx_l} \log \pt(\bx_l)} + \sqrt{ \color{violet} \eta} \bepsilon_l, \quad \bepsilon_l \sim \cN(0, \bI)
		\]
		\vspace{-0.5cm}
	\end{block}
    \eqpause
	\vspace{-0.7cm}
	\begin{multline*}
		\bx_t = \sqrt{1 - \beta_t}\, \bx_{t - 1} + \sqrt{\beta_t} \bepsilon_t \approx \left(1 - \frac{\beta_t}{2}\right)\bx_{t-1} + \sqrt{\beta_t} \bepsilon_t = \\
		= \bx_{t-1} + \frac{ \color{violet}  \beta_t}{2} {\color{teal} ( -\bx_{t-1} )} + \sqrt{ \color{violet}  \beta_t} \bepsilon_t
	\end{multline*}
    \eqpause
	\vspace{-0.7cm}
	\begin{itemize}
		\item ${\color{violet} \beta_t = \eta }$
		\item ${\color{teal} \nabla_{\bx_{t-1}}\log p(\bx_{t-1} | \btheta) = - \bx_{t-1} = \nabla_{\bx_{t-1}} \log \cN(0, \bI)}$
	\end{itemize}
 \end{frame}
%=======
\begin{frame}{Forward Gaussian Diffusion Process}
    \myfootnotewithlink{http://proceedings.mlr.press/v37/sohl-dickstein15.pdf}{Sohl-Dickstein J. Deep Unsupervised Learning using Nonequilibrium Thermodynamics, 2015}
	\[
		\bx_t = \sqrt{1 - \beta_t} \bx_{t-1} + \sqrt{\beta_t} \bepsilon_t, \;\; \bepsilon_t \sim \cN(0, \bI)
	\]
	\[
		q(\bx_t | \bx_{t-1}) = \cN(\sqrt{1 - \beta_t} \bx_{t-1}, \beta_t \bI)
	\]
    \eqpause
	\vspace{-0.5cm}
	\begin{block}{Statement 1}
		Let $\alpha_t = 1 - \beta_t$ and $\bar{\alpha}_t = \prod_{s=1}^t \alpha_s = \prod_{s=1}^t (1 - \beta_s)$. Then
		\[
			q(\bx_t | \bx_0) = \cN(\sqrt{\bar{\alpha}_t} \,\bx_0, (1 - \bar{\alpha}_t) \bI)
		\]
        \eqpause
		Thus, samples at any timestep $t$ can be generated directly from $\bx_0$
		\vspace{-0.2cm}
		{\small
		\begin{multline*}
			\bx_t = \sqrt{\alpha_t} {\color{teal}\bx_{t-1}} + \sqrt{1 - \alpha_t} \bepsilon_t
			\nextonslide{ = \\ = \sqrt{\alpha_t} ( {\color{teal} \sqrt{\alpha_{t-1}} \bx_{t-2} + \sqrt{1 - \alpha_{t-1}} \bepsilon_{t-1} } ) + \sqrt{1 - \alpha_t} \bepsilon_t}
			\nextonslide{ = \\ = \sqrt{\alpha_t \alpha_{t-1}} \bx_{t-2} + ( {\color{violet} \sqrt{\alpha_t (1-\alpha_{t-1})} \bepsilon_{t-1} + \sqrt{1-\alpha_t} \bepsilon_t } )}
			\nextonslide{ = \\ = \sqrt{\alpha_t \alpha_{t-1}} \bx_{t-2} + {\color{violet} \sqrt{1-\alpha_t \alpha_{t-1}} \bepsilon'_t}}
			\nextonslide{= \\ = \ldots = \sqrt{\bar{\alpha}_t}\, \bx_0 + \sqrt{1-\bar{\alpha}_t} \bepsilon, \quad \bepsilon \sim \cN(0, \bI)}
		\end{multline*}
		}
	\end{block}
 \end{frame}
%=======
\begin{frame}{Forward Gaussian Diffusion Process}
    \myfootnote{\href{https://arxiv.org/abs/2403.18103}{Chan S. Tutorial on Diffusion Models for Imaging and Vision, 2024} \\ \href{http://proceedings.mlr.press/v37/sohl-dickstein15.pdf}{Sohl-Dickstein J. Deep Unsupervised Learning using Nonequilibrium Thermodynamics, 2015}}
	\vspace{-0.3cm}
	{\small
	\[
		q(\bx_t | \bx_{t-1}) = \cN\left(\sqrt{1 - \beta_t} \bx_{t-1}, \beta_t \bI\right); \quad q(\bx_t | \bx_0) = \cN\left(\sqrt{\bar{\alpha}_t} \bx_0, (1 - \bar{\alpha}_t) \bI\right)
	\]
	}
	\vspace{-0.8cm}
	\begin{figure}
		\includegraphics[width=0.8\linewidth]{figs/conditional_diffusion}
	\end{figure}
    \eqpause
	\vspace{-0.5cm}
	\begin{block}{Statement 2}
		Applying the Markov chain to any distribution $\pd(\bx)$ yields $\bx_\infty \sim p_\infty(\bx) = \cN(0, \bI)$, the \textbf{stationary} (limiting) distribution:
		\[
			p_\infty(\bx) = \int q(\bx | \bx') p_\infty(\bx') d\bx' 
		\]

		\vspace{-0.5cm}
        \eqpause
		\[
			p_\infty(\bx) = \int q(\bx_\infty | \bx_0) \pd(\bx_0) d\bx_0 \approx \cN(0, \bI) \int \pd(\bx_0) d\bx_0 = \cN(0, \bI)
		\]
	\end{block}
 \end{frame}
%=======
\begin{frame}{Forward Gaussian Diffusion Process}
    \myfootnotewithlink{https://ayandas.me/blog-tut/2021/12/04/diffusion-prob-models.html}{Das A. An introduction to Diffusion Probabilistic Models, blog post, 2021}
	\textbf{Diffusion} describes the migration of particles from regions of high density to those of low density.
	\vspace{-0.2cm}
	\begin{figure}
		\includegraphics[width=0.5\linewidth]{figs/diffusion_over_time}
	\end{figure}
    \eqpause
	\vspace{-0.5cm}
	\begin{enumerate}
		\item $\bx_0 = \bx \sim \pd(\bx)$
		\item $\bx_t = \sqrt{1 - \beta_t} \bx_{t-1} + \sqrt{\beta_t} \bepsilon$, $\bepsilon \sim \cN(0, \bI)$, $t \geq 1$
		\item After $T \gg 1$ steps: $\bx_T \sim p_\infty(\bx) = \cN(0, \bI)$
	\end{enumerate}
    \eqpause
	If this process can be reversed, we can sample from $\pd(\bx)$ by starting from noise $p_\infty(\bx) = \cN(0, \bI)$. \\
	Our goal now becomes inverting this diffusion.
\end{frame}
%=======
\section{Denoising Score Matching for Diffusion}
%=======
\begin{frame}{Denoising Score Matching}
	\begin{block}{NCSN} 
		\vspace{-0.7cm} 
		\[
			q(\bx_t | \bx_0) = \cN(\bx_0, \sigma_t^2 \bI), \quad q(\bx_1) \approx \pd(\bx), \quad q(\bx_T) \approx \cN(0, \sigma_T^2 \bI)
		\]
		\[
			\nabla_{\bx_t} \log q(\bx_t | \bx) = - \frac{\bx_t - \bx}{\sigma_t^2}
		\]
		\vspace{-0.6cm} 
	\end{block}
    \eqpause
	\begin{block}{Gaussian Diffusion}
		\vspace{-0.7cm} 
		\[
			q(\bx_t | \bx_0) = \cN(\sqrt{\bar{\alpha}_t} \bx_0, (1 - \bar{\alpha}_t) \bI), \quad q(\bx_1) \approx \pd(\bx), \quad q(\bx_T) \approx \cN(0, \bI)
		\]
		\[
			\nabla_{\bx_t} \log q(\bx_t | \bx_0) = - \frac{\bx_t - \sqrt{\bar{\alpha}_t} \bx_0}{1 - \bar{\alpha}_t}
		\]
		\vspace{-0.6cm} 
	\end{block}
    \eqpause
	\begin{block}{Theorem (Denoising Score Matching)}
		\vspace{-0.7cm}
		\begin{multline*}
			\bbE_{q(\bx_t)}\left\| \bs_{\btheta, t}(\bx_t) - \nabla_{\bx_t} \log q(\bx_t) \right\|_2^2 = \\
			= \bbE_{\pd(\bx)} \bbE_{q(\bx_t | \bx)}\left\| \bs_{\btheta, t}(\bx_t) - \nabla_{\bx_t} \log q(\bx_t | \bx) \right\|_2^2 + \text{const}(\btheta)
		\end{multline*}
		\vspace{-0.5cm}
	\end{block}
    \eqpause
	\textbf{Note:} This enables applying the NCSN approach with annealed Langevin dynamics to diffusion-based denoising models.
\end{frame}
%=======
\section{Reverse Gaussian Diffusion Process}
%=======
\begin{frame}{Reverse Gaussian Diffusion Process}
    \myfootnotewithlink{https://lilianweng.github.io/posts/2021-07-11-diffusion-models/}{Weng L. What are Diffusion Models?, blog post, 2021}
	\begin{figure}
		\includegraphics[width=0.8\linewidth]{figs/DDPM}
	\end{figure}
    \eqpause
	\vspace{-0.5cm}
	\begin{block}{Forward Process}
		\vspace{-0.3cm}
		\[
			q(\bx_t | \bx_{t-1}) = \cN\left(\sqrt{1 - \beta_t} \bx_{t-1}, \beta_t \bI\right)
		\]
		\vspace{-0.5cm}
	\end{block}
    \eqpause
	\begin{block}{Reverse Process}
		\vspace{-0.3cm}
		\[
			q(\bx_{t-1}\mid\bx_{t}) = \frac{q(\bx_{t}\mid\bx_{t-1}) {\color{violet}q(\bx_{t-1})}}{{\color{violet}q(\bx_{t})}} 
			\approx \pt(\bx_{t - 1} | \bx_t)
		\]
        \eqpause
		\vspace{-0.3cm}
		${\color{violet}q(\bx_{t-1})}$ and ${\color{violet}q(\bx_{t})}$ are intractable:
		\[
			q(\bx_{t}) = \int q(\bx_{t} | \bx_0) \pd(\bx_0) d\bx_0
		\]
	\end{block}
\end{frame}
%=======
\begin{frame}{Reverse Gaussian Diffusion Process}
    \myfootnote{\href{}{Feller W. On the theory of stochastic processes, with particular reference to applications, 1949} \\ 
    \href{https://arxiv.org/abs/2112.07804}{Xiao Z., Kreis K., Vahdat A. Tackling the generative learning trilemma with denoising diffusion GANs, 2021}}
	\vspace{-0.4cm}
	\[
		q(\bx_{t-1}|\bx_{t}) = \frac{q(\bx_{t}|\bx_{t-1}) {\color{violet}q(\bx_{t-1})}}{{\color{violet}q(\bx_{t})}} 
	\]
	\vspace{-0.4cm}
	\begin{block}{Theorem (Feller, 1949)}
		If $\beta_t$ is sufficiently small, $q(\bx_{t-1}|\bx_t)$ is Gaussian {\color{gray}(thus, diffusion requires $T \approx 1000$ steps for convergence)}
	\end{block}
	\eqpause
	\vspace{-0.3cm}
	\begin{figure}
		\includegraphics[width=0.7\linewidth]{figs/inverse_distr_1d}
	\end{figure}
\end{frame} 
%=======
\begin{frame}{Reverse Gaussian Diffusion Process (Ancestral Sampling)}
    \myfootnotewithlink{https://lilianweng.github.io/posts/2021-07-11-diffusion-models/}{Weng L. What are Diffusion Models?, blog post, 2021}
	\vspace{-0.3cm} 
	\begin{figure}
		\includegraphics[width=0.8\linewidth]{figs/DDPM}
	\end{figure}
	\vspace{-0.3cm} 
	Define the reverse process as: 
	\vspace{-0.2cm}
	\[
		q(\bx_{t-1}|\bx_{t}) \approx \pt(\bx_{t - 1} | \bx_t) = \cN\bigl(\bmu_{\btheta, t}(\bx_t), \bsigma^2_{\btheta, t}(\bx_t)\bigr)
	\]
	{\color{gray}Feller's theorem justifies this Gaussian assumption.}
	\eqpause
	\begin{minipage}{0.5\linewidth}
		\begin{block}{Forward Process}
			\begin{enumerate}
				\item $\bx_0 = \bx \sim \pd(\bx)$
				\item $\bx_t = \sqrt{1 - \beta_t} \bx_{t-1} + \sqrt{\beta_t} \bepsilon$
				\item $\bx_T \sim p_\infty(\bx) = \cN(0, \bI)$
			\end{enumerate}
		\end{block}
	\end{minipage}%
    \eqpause
	\begin{minipage}{0.55\linewidth}
		\begin{block}{Reverse Process}
			\begin{enumerate}
				\item $\bx_T \sim p_\infty(\bx) = \cN(0, \bI)$
				\item $\bx_{t-1} = \bsigma_{\btheta, t}(\bx_t) \bepsilon + \bmu_{\btheta, t}(\bx_t)$
				\item $\bx_0 = \bx \sim \pd(\bx)$
			\end{enumerate}
		\end{block}
	\end{minipage}
    \eqpause
	\textbf{Note:} The forward process is non-learnable, i.e., it does not involve trainable parameters
\end{frame}
%=======
\begin{frame}{Summary}
	\begin{itemize}
		\item Denoising score matching minimizes the Fisher divergence on corrupted samples, making the divergence estimable via sampling
		\vfill 
		\item The noise-conditioned score network leverages a range of noise levels and annealed Langevin dynamics to learn the score function and enable sampling	
		\vfill
		\item The Gaussian diffusion process is a Markov chain that incrementally corrupts data with carefully structured Gaussian noise
		\vfill
		\item Denoising score matching, together with Langevin dynamics, can be applied to the Gaussian diffusion process
		\vfill
		\item The reverse process reconstructs data from noise samples, although its precise form is intractable
	\end{itemize}
\end{frame}
%=======
\end{document}